% vim: ts=4 sts=4 sw=4 et tw=75
\chapter{Style}
\label{chap:style}
\begin{quote}
    It is an old observation that the best writers sometimes disregard the
    rules of rhetoric(修辞学). When they do so, however, the reader will
    usually find in the sentence some compensating(补偿) merit(优点),
    attained at the cost of the violation(违反). Unless he is centain of
    doing as well, he will probaly do best to follow the rules.
\end{quote}

\begin{quotesrc}
    William Strunk and E. B. White, \bookname{The Elements of Style}
\end{quotesrc}

This fragment of code comes from a large program written many years ago:
\begin{wellcode}
    if ((country == SING) || (country == BRNI) ||
        (country == POL) || (country == ITALY))
    {
        /*
         * If the country is Singapore, Brunei or Poland
         * then the current time is the answer time 
         * rather than the off hook time.
         * Reset answer time and set day of week.
         */
         ...
    }
\end{wellcode}

It's carefully written, formatted, and commented, and the program it comes
from works extremly well; the programmers who create these system are
rightly proud of what they built. But this except is puzzling to the casual
reader. What relationship links Singapore, Brunei, Poland and Italy? Why
isn't Italy mentioned in the comment? Since the comment and the code
differ, one of them must be wrong. Maybe both are. The code is what gets
executed and tested, so it's more likely to be right; probably the comment
didn't get updated when the code did. The comment doesn't say enough about
the relationship among the three countries it does mention; if you had to
maintain this code, you would need to know more.

The few lines above are typical of much real code: mostly well done, but
with some things could be improved.

This book is about te practice of programming -- how to write programs for
real. Our purpose is to help you to write software that works at least as
well as the program this example was take from, while avoiding trouble
spots and weakness. We will talk about writing better code from the
beginning and improving it as it evolves.

We are going to start in an unusual place, however, by discussing
programming style. The purpose of style is to make the code easy to read
for yourself and others, and good style is curcial to good programming. We
want to talk about it first so you will be sensitive to it as you read the
code in the rest of the book.

These is more to writing a program than getting the syntax right, fixing
the bugs, and making it run fast enough. Programs are read not only by
computers but also by programmers. A well-written program is easier to
understand and to modify than a poorly-written one. The discipline of
writing well leads to code that is more like to be correct. Fortunately,
this discipline is not hard.

The principles of programming style are based on common sense guided
experience, not on arbitrary rules and prescriptions(命令). Code should be
clear and simle -- straightforward login, natural expression, conventional
language use, meaningful names, neet formatting, helpful comments -- and it
should be avoid clever tricks and unusual constructions. Consitency is
important because others will find it easier to read your code, and you
theirs, if you all stick to the same style. Detail many be imposed by local
conventions, management edict(法令), or a program, but even if not, but if
even not, it is best to obey a set of widely shared conventions. We follow
the style used in the book \bookname{The C Programming Language}, with
minor adjustments for C++ and Java.

We will often illustrate rules of style by small example of bad and good
programming, since the contrast between two ways of saying the same thing
is instructive. These example are not artifical. The "bad" ones are all
adapted from real code, written by ordinary programmers(occasionally
yourselves) working under the common pressures of too much work and too
little time. Some will be distilled(提取) for brevity(简短), but they will
not be misrepresented(错误的叙述). Then we will rewrite the bad
excerpts(摘录) to show how they could be improved. Since they are real
code, however, they may exhibit(展现) multiple problems. Addressing every
shortcoming would take us too far off topics, so some of the good examples
will still harbor(隐藏) other, unremarked flaws(缺点).

To distiguish bad examples from good, throughout the book we will place
question marks in the questionable code, as in this real excerpt(摘录):
\begin{badcode}
    #define ONE 1
    #define TEN 10 
    #define TWENTY  20
\end{badcode}
Why are these \verb"#define"s questionable? Consider the modification that
will be necessary if an array of \verb"TWENTY" elements must be made
larger. At the very least(至少), each name should be replaced by one that
indicates the role of the specific value in the program:
\begin{wellcode}
    #define INPUT_MODE  1
    #define INPUT_BUFSIZE   10
    #define OUTPUT_BUFSIZE  20 
\end{wellcode}

\section{Names}
\label{sec:names}
What's in a name? A variable or function name labels an object and conveys
information about its purpose. A name should be informative, concise,
memorable, and pronounceable if possible. Much information coomes from
context and scope; the broader the scope of a variable, the more
information should be conveyed by it's name.

\emph{Use descriptive names for globals, short names for locals.} Global
variables, by definition, can crop up(突然出现) anywhere in a program, so
they need names long enough and descriptive enough to remind the reader of
their meaning. It's also helpful to include a brief comment with the
declaration of each global:
\begin{wellcode}
    int npending = 0;   // current length of input queue
\end{wellcode}
Global functions, classes and structures should also have descriptive names
that suggest their role in a program.

By contrast, shorter names suffice(足够) for local variables; within a
function, \verb"n" may be sufficient, \verb"npoints" is fine, and
\verb"numberOfPoints" is overkill.

Local variables used in conventional way can have very short names. The use
of \verb"i" and \verb"j" for loop indices, \verb"p" and \verb"q" for
pointers, and \verb"s" and \verb"t" for strings is so frequent that there
is little profit and perhaps some loss in longer names. Compare
\begin{badcode}
    for (theElementIndex = 0; theElementIndex < numberOfElements;
            theElementIndex++)
        elementArray[theElementIndex] = theElementIndex;
\end{badcode}
to 
\begin{wellcode}
    for (i = 0; i < nelems; i++)
        elem[i] = i;
\end{wellcode}
Programmers are often encouraged to use long variable names regardless of
context. That is s mistake: clarity is often achieved through brevity.

There are many naming convensions and local customs. Common ones include
using names that begin or end with \verb"p", such as \verb"nodep", for
pointer; initial capital letters for \verb"Global"s; and all capital for
\verb"CONSTANT"s. Some programming shop use more sweeping(彻底的) rules,
such as notation to encode type and usage information in the variable,
perhaps \verb"pch" to mean a pointer to a character and \verb"strTo" and
\verb"strFrom" to mean strings that will be written to and read from. As
for the spelling of the names themselves, whether to use \verb"npending" or
\verb"numPending" or \verb"num_pending" is a matter of taste; specific
rules are much less important than consistent adherence(坚持) to a sensible
convention.

Naming conventions make it easier to understand your own code, as well as
code written by others. They also make it easier to invent new names as the
code is being written. The longer the program, the more important is the
choice of good, descriptive, systematic names.

Namespaces in C++ and packages in Java provide ways to manage the scope of
names and help to keep meanings clear without unduly(过度的) long names.

\emph{Be consistent.} Give related things related names that show their
relationshipt and highlight their difference.

Besides beging much too long, the member names in this Java class are
wildly(鲁莽地) inconsistent:
\begin{badcode}
    class UserQueue {
        int noOfItemsInQ, frontOfTheQueue, queueCapacity;
        public int noOfUserInQueue() { ... }
    }
\end{badcode}
The word "queue" appears as Q, Queue and queue. But since queues can only
be accessed from a variable of type \verb"UserQueue", member names do not
need to mention "queue" at all; context suffices, so 
\begin{badcode}
    queue.queueCapacity
\end{badcode}
is redundant. This version is better:
\begin{wellcode}
    class UserQueue {
        int nitems, front, capacity;
        public int nusers() { ... }
    }
\end{wellcode}
since it leads to statements like 
\begin{wellcode}
    queue.capacity++;
    n = queue.nusers();
\end{wellcode}
No clarity is lost. This example still needs work, however, "items" and
"users" are the same thing, so only one term should be used for a single
concept.

\emph{Use active names for functions.} Function names should be based on
active verbs, perhaps followed by nouns:
\begin{wellcode}
    now = data.getTime();
    putchar('\n');
\end{wellcode}
Functions that return boolean(true or false) value should be named so that
return value is unambiguous. Thus
\begin{badcode}
    if (checkoctal(c)) ...
\end{badcode}
does not indicate which value is true and which is false, while 
\begin{wellcode}
    if (isoctal(c)) ...
\end{wellcode}
makes it clear that the function return true if argument is octal and false
if not.

\emph{Be accurate.} A name not only labels, it conveys information to the
reader. A misleading names can result in mystifying(模糊的) bugs.

One of us wrote and distributed for years a macro called \verb"isoctal"
with this incorrect implementation:
\begin{badcode}
    #define isoctal(c) ((c) >= '0' && (c) <= '8')
\end{badcode}
instead of the proper
\begin{wellcode}
    #define isoctal(c) ((c) >= '0' && (c) <= '7')
\end{wellcode}
In this case, the name conveyed the correct intent but the implementation
was wrong; it's easy for a sensible name to disguise a broken
implementation.

Here is an example in which the name and the code are in complete
contradiction(矛盾):
\begin{badcode}
    public boolean inTable(Object obj) {
        int j = this.getIndex(obj);
        return(j == nTable);
    }
\end{badcode}

The function \verb"getIndex" returns a value between zero and
\verb"nTable-1" if it finds the object, and returns \verb"nTable" if not.
The boolean value returned by \verb"inTable" is thus the oppsite of what
the name implies. At the time code is written, this might not cause
trouble, but it the program is modified later, perhaps by a different
programmer, the name is sure to confuse.

\par
\exercise Comment on the choice of names and values in the following code.
\begin{badcode}
    #define TRUE    0
    #define FALSE   1

    if ((ch = getchar()) == EOF)
        not_eof = FALSE;
\end{badcode}

\exercise Improve this function:
\begin{badcode}
    int smaller(char *s, char *t) {
        if (strcmp(s, t) < 1)
            return 1;
        else 
            return 0;
    }
\end{badcode}

\exercise Read this code aloud:
\begin{badcode}
    if ((falloc(SMRHSHSCRTCH, S_IFEXT|0644, MAXRODDHSH)) < 0)
        ...
\end{badcode}

\section{Expression and Statements}
\label{sec:exprstat}
By analogy(类比) with choosing names to aid the reader's understanding,
write expressions and statements in a way that makes their meaning as
transport as possible. Write the clearest code that does the job. Use
spaces around operators to suggest grouping; more generally, format to help
readability. This is trivial(琐细的) but valuable, like keeping a neat dest
so you can find things. Unlike your dest, your programs are likely bo be
examined by others.

\emph{Indent to show structure.} A consistent indentation style is the
lowest-energy way to make a program's structure self-evident(不言自明的).
This example is badly formatted:
\begin{badcode}
    for(n++;n<100;field[n++]='\0');
    *i = '\0'; return('\n');
\end{badcode}
Reformatting improves it somewhat:
\begin{badcode}
    for (n++; n < 100; field[n++] = '\0')
        ;
    *i = '\0';
    return('\n');
\end{badcode}
Even better is to put the assignment in the body and separate the
increment, so the loop takes a more conventional form and is thus easier to
grasp(抓取):
\begin{wellcode}
    for (n++; n < 100; n++)
        field[n] = '\0';
    *i = '\0';
    return '\n';
\end{wellcode}
\emph{Use the natural form for expressions.} Write expressions as you might
speak them aloud. Conditional expressions that include negations are always
hard to understand:
\begin{badcode}
    if (!(block-id < actblks) || !(block-id >= unblocks))
        ...
\end{badcode}
Each test is stated negatively, though there is no need for either to be.
Turing the relations around lets us state the tests positively:
\begin{wellcode}
    if ((block-id >= actblks) || (block - id < unblocks))
        ...
\end{wellcode}
Node the code reads naturally.
\emph{Parenthesize to resolves ambiguity.} Parentheses specify grouping and
can be used to make the intent clear event when they are not required. The
inner parentheses in the previous example are not necessary, but they don't
hurt, either. Seasoned(经验丰富的) programmers might omit them, because the
relational operators(\verb"< <= == != >= >") have higher precedence than
the locgical operators(\verb"&&" and \verb"||").

When mixing unrelated oprators, though, it's a good idea to parenthesize. C
and it's friends present pernicious(恶劣的) precedence problems, and it's
easy to make a mistake. Because the logical operators bind tighter than
assignment, parentheses are mandatory for most expressions that combine them:
\begin{wellcode}
    while ((c = getchar()) != EOF)
        ...
\end{wellcode}
The bitwise operators \verb"&" and \verb"|" have lower precedence than
relational operators like \verb"==", so despite it's appearance,
\begin{badcode}
    if (x&MASK == BITS)
        ...
\end{badcode}
actually means
\begin{badcode}
    if (x & (MASK==BITS))
        ...
\end{badcode}
which is centainly not the programmer's intent. Because it combines bitwise
and relational operators, the expression need parentheses:
\begin{wellcode}
    if ((x&MASK) == BITS)
        ...
\end{wellcode}
Even if parentheses aren't necessary, they can help if the grouping is hard
to grasp at first glance. This code doesn't need parentheses:
\begin{badcode}
    leap_year = y % 4 == 0 && y % 100 != 0 || y % 400 == 0;
\end{badcode}
but they make it easier to understand:
\begin{wellcode}
    leap_year = ((y%4 == 0) && (y%100 != 0)) || (y%400 == 0);
\end{wellcode}
We also removed some of the blanks: grouping the operands of
higher-precendence operators helps the readers to see the structure more
quickly.

\emph{Break up complex expressions.} C, C++ and Java have rich expression
syntax and operators, and it's easy to get carried away by cramming(塞满)
everything into one construction. An expression like the following is
compact(紧凑的) but it packs too many oprations into a single statement:
\begin{badcode}
    *x += (*xp=(2*k < (n-m) ? c[k+1] : d[k--]));
\end{badcode}
It's easier to grasp when broken into several pieces:
\begin{wellcode}
    if (2*k < n-m)
        *xp = c[k+1];
    else 
        *xp = d[k--];
    *x += *xp;
\end{wellcode}
\emph{Be clear.} Programmers' endless creative energy is sometimes used to
write the most concise code possible, or to find clever ways to achieve a
result. Sometimes these skills are misapplied, though, since the goal is to
write clear code, not clever code.

What does this intricate(复杂的) calculation do?
\begin{badcode}
    subkey = subkey >> (bitoff - ((bitoff >> 3) << 3));
\end{badcode}
The innermost expression shifts \verb"bitoff" three bits to the right. The
result is shifted left again, thus replacing the three shifted bits by
zeros. This result in turn is subtracted from the original value, yielding
the bottom three bits of \verb"bitoff". These three bits are used to shift
\verb'subkey' to the right.

Thus the original expression is equivalent to 
\begin{wellcode}
    subkey = subkey >> (bitoff & 0x7);
\end{wellcode}
It takes a while to puzzle out what the first version is doing; the second
is shorter and clearer. Experienced programmers make it ever shorter by
using assignment operator:
\begin{wellcode}
    subkey >>= bitoff & 0x7;
\end{wellcode}
Some constructs seem to invite abuse. The \verb"?:" operator can lead to
mysterious code:
\begin{badcode}
    child=(!LC&&!RC)?0:(!LC?RC:LC);
\end{badcode}
It's almost impossible to figure out what this code without following all
the possible paths through the expression. This form is longer, but much
easier to follow because it makes the paths explicit:
\begin{wellcode}
    if (LC == 0 && RC == 0)
        child = 0;
    else if (LC == 0)
        child = RC;
    else
        child = LC;
\end{wellcode}
The \verb'?:' operator is fine for short expressions where it can replace
four lines of if-else with one, as in
\begin{wellcode}
    max = (a > b) ? a : b;
\end{wellcode}
or perhaps
\begin{wellcode}
    printf("The list has %d item%s\n", n, n==1 ? "" : "s");
\end{wellcode}
but it is not a general replacement for conditional statements.

\emph{Be careful with side effect.} Operators like \verb'++' have side
effects: besides returning a value, they also modify an underlying
variable. Side effects can be extremely convenient, but they can also cause
trouble because the actions of retrieving the value and updating the
variable might not happen at the same time. In C and C++, the order of
execution of side effects is undefined, so this multiple assignment is
likely to produce the wrong answer:
\begin{badcode}
    str[i++] = str[i++] = ' ';
\end{badcode}
The intent is to store blanks at the next two position in \verb'str'. But
depending on when \verb'i' is updated, a position in \verb'str' could be
skipped and \verb'i' might end up increased only by 1. Break it into two
statements:
\begin{wellcode}
    str[i++] = ' ';
    str[i++] = ' ';
\end{wellcode}

Even though it cantains only one increment, this assignment can also give
varying results:
\begin{badcode}
    array[i++] = i;
\end{badcode}
If \verb'i' is initially 3, the array element might be set to 3 or 4.

It's not just increment and decrement that have side effects; I/O is
another source of behind-the-scenes(幕后的) action. This example is an
attempt to read two related numbers from standard input:
\begin{badcode}
    scanf("%d %d", &yr, &profit[yr]);
\end{badcode}
It is broken because part of the expression modifies \verb'yr' and another
part uses it. The value of \verb'profit[yr]' can never be right unless the
new value of \verb'yr' is the same as the old one. You might think that the
answer depends on the order in which the arguments are evaluated, but the
real issue is that all the arguments to \verb'scanf' are evaluated before
the routine is called, so \verb'&profit[yr]' will always be evaluated using
the old value of \verb'yr'. This sort of problem can occur in almost any
language. The fix is, as usual, to break up the expression:
\begin{wellcode}
    scanf("%d", &yr);
    scanf("%d", &profit[yr]);
\end{wellcode}
Exercises cantion in any expression with side effects.

\par
\exercise Improves each of these fragments:
\begin{badcode}
    if (!(c == 'y' || c == 'Y'))
        return;
    length = (length < BUFSIZE) ? length : BUFSIZE;
    flag = flag ? 0 : 1;
    quote = (*line == '"') ? 1 : 0;
    if (val & 1)
        bit = 1;
    else
        bit = 0;
\end{badcode}

\exercise What is wrong with this excerpt(摘录)?
\begin{badcode}
    int read(int *ip) {
        scanf("%d", ip);
        return *ip;
    }
        ...
    insert(&graph[vert], read(&val), read(&ch));
\end{badcode}

\exercise List all the different output this could produce with various
orders of evaluation:
\begin{badcode}
    n = 1;
    printf("%d %d\n", n++, n++);
\end{badcode}
Try it on as many compilers as you can, to see what happens in practice.

