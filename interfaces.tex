% vim: ts=4 sts=4 sw=4 et tw=75
\chapter{Interfaces}
\label{chap:interface}
\begin{quote}
    Before I build a wall I'd ask to know \\
    What I was walling in or walling out, \\
    And to whom I was like to give offense. \\
    Something there is that doesn't love a wall. \\
    That wants it down.
\end{quote}
\begin{quotesrc}
    Robert Frost, \bookname{Mending Wall}
\end{quotesrc}

The essence (精髓) of design is to balance competing goals and constraints.
Although there may be many tradeoffs when one is writing a small
self-contained system, the ramifications (分叉) of particular choices
remain within the system and affect only the individual programmer. But
when code is to be used by others, decisions have wider repercussions
(反响).

Among the issues to be worked out in a design are
\begin{itemize}
    \item Interfaces: what services and access are provided? The interface
        is in effect a contract between supplier and customer. The desire
        is to provide services that are uniform and convenient, with enough
        functionality to be easy to use but not so much as to become
        unwieldy (笨拙).
    \item Information hiding: what information is visible and what is
        private? An interface must provide straightforward access to the
        components while hiding details of the implementation so they can
        be changed without affecting users.
    \item Resource management: who is responsible for managing memory and
        other limited resources? Here, the main problems are allocating and
        freeing storage, and managing shared copies of information.
    \item Error handling: who detects errors, who reports them, and how?
        When an error is detected, what recovery is attempted?
\end{itemize}

In Chapter \ref{chap:alds} we looked at the individual pieces -- the data
structures -- from which a system is built. In Chapter \ref{chap:desipl},
we looked at how to combine those into a small program. The topic now turns
to the interfaces between components that might come from different
sources. In this chapter we illustrate interface design by building a
library of functions and data structures for a common task. Along the way,
we will present some principles of design. Typically there are an enormous
number of decisions to be made, but most are made almost unconsciously.
Without these principles, the result is often the sort of haphazard
(无计划的) interfaces that frustrate and impede (妨碍) programmers every
day.

\section{Comma-Separated Values}
\label{sec:comma_separated_values}

\emph{Comma-separated values}, or \emph{CSV}, is the term for a natural and
widely used representation for tabular (表格式的) data. Each row of a table
is a line of text; the fields on each line are separated by commas. The
table at the end of the previous chapter might begin this way in CSV
format: \\
\indent\indent ,"250MHz","400MHz","Line of" \\
\indent\indent ,"R10000","Pentium II","source code" \\
\indent\indent C,0.36 sec,0.30 sec,150 \\
\indent\indent Java,4.9,9.2,105

This format is read and written by programs such as spreadsheets; not
coincidentally (巧合), it also appears on web pages for services such as
stock price quotations. A popular web page for stock quotes presents a
display like this:
\begin{figure}[h]
    \centering
\begin{tabular}{|c|c|c|c|c|c|}
    \hline
    \textbf{Symbol}  & \multicolumn{2}{|c|}{\textbf{Last Trade}} &
    \multicolumn{2}{|c|}{\textbf{Change}} & \textbf{Volume}   \\
    \hline
    LU  & 2:19PM    & 86-114    & +4-1/16   & +4.94\%   & 5,804,800 \\
    \hline
    T   & 2:19PM    & 60-11/16  & -1-3/16   & -1.92\%   & 2,468,000 \\
    \hline
    MSFT& 2:24PM    & 106-9/16  & +1-3/8    & +1.31\%   & 11,474,900\\
    \hline
\end{tabular}
\\
\vspace{1em}
{\centering\underline{Download spreadsheet Format}}
\end{figure}

Retrieving numbers by interacting with a web browser is effective but
time-consuming. It's a nuisance (讨厌的) to invoke a browser, wait, watch a
barrage (弹幕) of advertisements, type a list of stocks, wait, wait, wait,
then watch another barrage, all to get a few numbers. To process the
numbers further requires even more interaction; selecting the "Download
Spreadsheet Format" link retrieves a file that contains much the same
(几乎相同) information in lines of CSV data like these (edited to fit):
\begin{wellcode}
    "LU",86.25,"11/4/1998","2:19PM",+4.0625,
            83.9375,86.875,83.625,5804800
    "T",60.6875,"11/4/1998","2:19PM",-1.1875,
            62.375,62.625,60.4375,2468000
    "MSFT",106.5625,"11/4/1998","2:24PM",+1.375,
            105.8125,107.3125,105.5625,11474900
\end{wellcode}

Conspicuous (显著的) by its absence in this process is the principle of
letting the machine do the work. Browsers let your computer access data on
a remote server, but it would be more convenient to retrieve the data
without forced interaction. Underneath (在..下面) all the button-pushing is
a purely textual procedure -- the browser reads some HTML, you type some
text, the browser sends that to a server and reads some HTML back. With the
right tools and language, it's easy to retrieve the information
automatically.  Here's a program in the language Tcl to access the stock
quote web site and retrieve CSV data in the format above, preceded (开始)
by a few header lines:
\begin{wellcode}
    # getquotes.tcl: stock prices for Lucent, AT&T, Microsoft
    set so [socket quote.yahoo.com 80]  ;# connect to server
    set q "/d/quotes.csv?s=LU+T+MSFT&f=sl1d1t1ohgv"
    puts $so "GET $q HTTP/1.0\r\n\r\n"  ;# send request
    flush $so
    puts [read $so]                     ;# read & print reply
\end{wellcode}
The cryptic sequence \verb'f=...' that follows the ticker symbols is an
undocumented control string, analogous to the first argument of
\verb'printf', that determines what values to retrieve. By experiment, we
determined that \verb's' identifies the stock symbol, \verb'l1' the last
price, \verb'c1' the change since yesterday, and so on. What's important
isn't the details, which are subject to (可以) change anyway, but the
possibility of automation: retrieving the desired information and
converting it into the form we need without any human intervention. We can
let the machine do the work.

It typically takes a fraction of a second to run \verb'getquotes' , far
less than interacting with a browser. Once we have the data, we will want
to process it further, Data formats like CSV work best if there are
convenient libraries for converting to and from the format, perhaps allied
with (联合) some auxiliary processing such as numerical conversions. But we
do not know of an existing public library to handle CSV, so we will write
one ourselves.

In the next few sections, we will build three versions of a library to read
CSV data and convert it into an internal representation. Along the way,
we'll talk about issues that arise when designing software that must work
with other software. For example, there does not appear to be a standard
definition of CSV, so the implementation cannot be based on a precise
specification, a common situation in the design of interfaces.

\section{A Prototype Library}
\label{sec:a_prototype_library}

We are unlikely to get the design of a library or interface right on the
first attempt.  As Fred Brooks once wrote, "plan to throw one away; you
will, anyhow." Brooks was writing about large systems but the idea is
relevant for any substantial (有内容的) piece of software. It's not usually
until you've built and used a version of the program that you understand
the issues well enough to get the design right.

In this spirit, we will approach (接近) the construction of a library for
CSV by building one to throw away, a prototype. Our first version will
ignore many of the difficulties of a thoroughly engineered library, but
will be complete enough to be useful and to let us gain some familiarity
with the problem.

Our starting point is a function \verb'csvgetline' that reads one line of
CSV data from a file into a buffer, splits it into fields in an array,
removes quotes, and returns the number of fields. Over the years, we have
written similar code in almost every language we know, so it's a familiar
task. Here is a prototype version in C; we've marked it as questionable
because it is just a prototype:
\begin{badcode}
    char    buf[200];   /* input line buffer */
    char    *field[20]; /* fields */

    /* csvgetline: read and parse line, return field count */
    /* sample input: "LU",86.25,"11/4/1998","2:19PM",+4.0625 */
    int csvgetline(FILE *fin)
    {
        int     nfield;
        char    *p, *q;

        if (fgets(buf, sizeof(buf), fin) == NULL)
            return -1;
        nfield = 0;
        for (q = buf; (p=strtok(q, ",\n\r")) != NULL; q = NULL)
            field[nfield++] = unquote(p);
        return nfield;
    }
\end{badcode}
The comment at the top of the function includes an example of the input
format that the program accepts; such comments are helpful for programs
that parse messy (散乱的) input.

The CSV format is too complicated to be parsed easily by \verb'scanf' so we
use the C standard library function \verb'strtok'. Each call of
\verb'strtok(p, s)' returns a pointer to the first token within \verb'p'
consisting of characters not in \verb's'; \verb'strtok' terminates the
token by overwriting the following character of the original string with a
null byte. On the first call, \verb'strtok''s first argument is the string
to scan; subsequent calls use \verb'NULL' to indicate that scanning should
resume where it left off (停止) in the previous call.  This is a poor
interface. Because \verb'strtok' stores a variable in a secret place
between calls, only one sequence of calls may be active at one time;
unrelated interleaved (交错的) calls will interfere with each other.

Our function \verb'unquote' removes the leading and trailing quotes that
appear in the sample input above. It does not handle nested quotes,
however, so although sufficient for a prototype, it's not general.
\begin{badcode}
    /* unquote: remove leading and trailing quote */
    char *unquote(char *p)
    {
        if (p[0] == '"') {
            if (p[strlen(p)-1] == '"')
                p[strlen(p)-1] = '\0';
            p++;
        }
        return p;
    }
\end{badcode}

A simple test program helps verify that \verb'csvgetline' works:
\begin{badcode}
    /* csvtest main: test csvgetline function */
    int main(void)
    {
        int i, nf;

        while ((nf = csvgetline(stdin)) != -1)
            for (i = 0; i < nf; i++)
                printf("field[%d] = '%s'\n", i, field[i]);;
            return 0;
    }
\end{badcode}
The \verb'printf' encloses the fields in matching single quotes, which
demarcate (区分) them and
help to reveal bugs that handle white space incorrectly.

We can now run this on the output produced by \verb'getquotes.tcl':
\begin{wellcode}
    % getquotes.tcl | csvtest
    ...
    field[0] = 'LU'
    field[1] = '86.375'
    field[2] = '11/5/1998'
    field[3] = '1:01PM'
    field[4] = '-0.125'
    field[5] = '86'
    field[6] = '86.375'
    field[7] = '85.0625'
    field[8] = '2888600'
    field[0] = 'T'
    field[1] = '61.0625'
    ...
\end{wellcode}
(We have edited out the HTTP header lines.)

Now we have a prototype that seems to work on data of the sort (种类) we
showed
above.  But it might be prudent (谨慎的) to try it on something else as
well, especially if we plan to let others use it. We found another web site
that downloads stock quotes and obtained a file of similar information but
in a different form: carriage returns (\verb'\r') rather than newlines to
separate records, and no terminating carriage return at the end of the
file.  We've edited and formatted it to fit on the page:
\begin{wellcode}
    "Ticker","Price","Change","Open","Prev Close","Day High",
        "Day Low","52 Week High","52 Week Low","Dividend",
        "Yield","Volume","Average Volume","P/E"
    "LU",86.313,-0.188,86.000,86.500,86.438,85.063,108.50,
        36.18,0.16,0.1,2946700,9675000,N/A
    "T",61,125,0.938,60.375,60.188,61.125,60.000,68.50,
        46.50,1.32,2.1,3061000,4777000,17.0
    "MSFT",107.000,1.500,105.313,105.500,107.188,105.250,
        119.62,59.00,N/A,N/A,7977300,16965000,51.0
\end{wellcode}
With this input, our prototype failed miserably (非常不幸地).

We designed our prototype after examining one data source, and we tested it
originally only on data from that same source. Thus we shouldn't be
surprised when the first encounter with a different source reveals gross
(恶劣的) failings. Long input lines, many fields, and unexpected or missing
separators all cause trouble. This fragile prototype might serve for
personal use or to demonstrate the feasibility of an approach, but no more
than that. It's time to rethink the design before we try another
implementation.

We made a large number of decisions, both implicit and explicit, in the
prototype. Here are some of the choices that were made, not always in the
best way for a general-purpose library. Each raises an issue that needs
more careful attention.
\begin{itemize}
    \item The prototype doesn't handle long input lines or lots of fields.
        It can give wrong answers or crash because it doesn't even check
        for overflows, let alone return sensible values in case of errors.
    \item The input is assumed to consist of lines terminated by newlines.
    \item Fields are separated by commas and surrounding quotes are
        removed. There is no provision (准备) for embedded quotes or
        commas.  The input line is not preserved; it is overwritten by the
        process of creating fields.
    \item No data is saved from one input line to the next: if something is
        to be remembered, a copy must be made.
    \item Access to the fields is through a global variable, the
        \verb'field' array, which is shared by \verb'csvgetline' and
        functions that call it; there is no control over access to the
        \verb'field' contents or the pointers. There is also no attempt to
        prevent access beyond the last field.
    \item The global variables make the design unsuitable for a
        multi-threaded environment or even for two sequences of interleaved
        calls.
    \item The caller must open and close files explicitly;
        \verb'csvgetline' reads only from open files.
    \item Input and splitting are inextricably (无法避免地) linked: each
        call reads a line and splits it into fields, regardless of whether
        the application needs that service.
    \item The return value is the number of fields on the line; each line
        must be split to compute this value. There is also no way to
        distinguish errors from end of file.
    \item There is no way to change any of these properties without
        changing the code.
\end{itemize}

This long yet incomplete list illustrates some of the possible design
tradeoffs.  Each decision is woven through the code. That's fine for a
quick job, like parsing one fixed format from a known source. But what if
the format changes, or a comma appears within a quoted string, or the
server produces a long line or a lot of fields?

It may seem easy to cope (应付), since the "library" is small and only a
prototype anyway. Imagine, however, that after sitting on the shelf
(被搁置的) for a few months or years the code becomes part of a larger
program whose specification changes over time. How will \verb'csvgetline'
adapt? If that program is used by others, the quick choices made in the
original design may spell (招致) trouble that surfaces years later. This
scenario (情节) is representative of the history of many bad interfaces. It
is a sad fact that a lot of quick and dirty code ends up in widely-used
software, where it remains dirty and often not as quick as it should have
been anyway.

\section{A Library for Others}
\label{sec:a_library_for_others}

Using what we learned from the prototype, we now want to build a library
worthy of general use. The most obvious requirement is that we must make
\verb'csvgetline' more robust so it will handle long lines or many fields;
it must also be more careful in the parsing of fields.

To create an interface that others can use, we must consider the issues
listed at the beginning of this chapter: interfaces, information hiding,
resource management, and error handling. The interplay (互相作用) among
these strongly affects the design. Our separation of these issues is a bit
arbitrary, since they are interrelated.

\emph{Interface.}We decided on three basic operations: \\
\indent\indent\texttt{char *csvgetline(FILE *)}: read a new CSV line \\
\indent\indent\texttt{char *csvfield(int n)}: return the \verb'n'-th field
of the current line \\
\indent\indent\texttt{int csvnfield(void)}: return the number of fields on
the current

What function value should \verb'csvgetline' return? It is desirable to
return as much useful information as convenient, which suggests returning
the number of fields, as in the prototype. But then the number of fields
must be computed even if the fields aren't being used. Another possible
value is the input line length, which is affected by whether the trailing
newline is preserved. After several experiments, we decided that
\verb'csvgetline' will return a pointer to the original line of input, or
\verb'NULL' if end of file has been reached.

We will remove the newline at the end of the line returned by
\verb'csvgetline', since it can easily be restored if necessary.

The definition of a field is complicated; we have tried to match what we
observe empirically (经验的) in spreadsheets and other programs. A field is
a sequence of zero or more characters. Fields are separated by commas.
Leading and trailing blanks are preserved. A field may be enclosed in
double-quote characters, in which case it may contain commas. A quoted
field may contain double-quote characters, which are represented by a
doubled double-quote; the CSV field \verb'"x""y"' defines the string
\verb'x"y'.  Fields may be empty; a field specified as "" is empty, and
identical to one specified by adjacent commas.

Fields are numbered from zero. What if the user asks for a non-existent
field by calling \verb'csvfield(-1)' or \verb'csvfield(100000)'? We could
return \verb'""' (the empty string) because this can be printed and
compared; programs that process variable numbers of fields would not have
to take special precautions (预防) to deal with non-existent ones. But that
choice
provides no way to distinguish empty from non-existent. A second choice
would be to print an error message or even abort; we will discuss shortly
why this is not desirable. We decided to return \verb'NULL', the
conventional value for a non-existent string in C.

\emph{Information hiding.} The library will impose no limits on input line
length or number of fields. To achieve this, either the caller must provide
the memory or the callee (the library) must allocate it. The caller of the
library function \verb'fgets' passes in an array and a maximum size. If the
line is longer than the buffer, it is broken into pieces.  This behavior is
unsatisfactory for the CSV interface, so our library will allocate memory
as it discovers that more is needed.

Thus only \verb'csvgetline' knows about memory management; nothing about
the way that it organizes memory is accessible from outside. The best way
to provide that isolation is through a function interface:
\verb'csvgetline' reads the next line, no matter how big,
\verb'csvfield(n)' returns a pointer to the bytes of the \verb'n'-th field
of the current line, and \verb'csvnfield' returns the number of fields on
the current line.

We will have to grow memory as longer lines or more fields arrive. Details
of how that is done are hidden in the csv functions; no other part of the
program knows how this works, for instance whether the library uses small
arrays that grow, or very large arrays, or something completely different.
Nor does the interface reveal when memory is freed.

If the user calls only \verb'csvgetline', there's no need to split into
fields; lines can be split on demand. Whether field-splitting is eager
(热切的)
(done right away when the line is read) or lazy (done only when a field or
count is needed) or very lazy (only the requested field is split) is
another implementation detail hidden from the user.

\emph{Resource management.} We must decide who is responsible for shared
information.  Does \verb'csvgetline' return the original data or make a
copy? We decided that the return value of \verb'csvgetline' is a pointer to
the original input, which will be overwritten when the next line is read.
Fields will be built in a copy of the input line, and \verb'csvfield' will
return a pointer to the field within the copy. With this arrangement, the
user must make another copy if a particular line or field is to be saved or
changed, and it is the user's responsibility to release that storage when
it is no longer needed.

Who opens and closes the input file? Whoever opens an input file should do
the corresponding close: matching tasks should be done at the same level or
place. We will assume that \verb'csvgetline' is called with a \verb'FILE'
pointer to an already-open file that the caller will close when processing
is complete.

Managing the resources shared or passed across the boundary between a
library and its callers is a difficult task, and there are often sound
(合理的) but conflicting reasons to prefer various design choices. Errors
and misunderstandings about the shared responsibilities are a frequent
source of bugs.

\emph{Error handling.} Because \verb'csvgetline' returns \verb'NULL', there
is no good way to distinguish end of file from an error like running out of
memory; similarly, access to a non-existent field causes no error. By
analogy with \verb'ferror', we could add another function
\verb'csvgeterror' to the interface to report the most recent error, but
for simplicity we will leave it out of this version.

As a principle, library routines should not just die when an error occurs;
error status should be returned to the caller for appropriate action. Nor
should they print messages or pop up dialog boxes, since they may be
running in an environment where a message would interfere with something
else. Error handling is a topic worth a separate discussion of its own,
later in this chapter.

\emph{Specification.} The choices made above should be collected in one
place as a specification of the services that \verb'csvgetline' provides
and how it is to be used.  In a large project, the specification precedes
the implementation, because specifiers and implementers are usually
different people and may be in different organizations.  In practice,
however, work often proceeds in parallel, with specification and code
evolving together, although sometimes the "specification" is written only
after the fact to describe approximately what the code does.

The best approach is to write the specification early and revise it as we
learn from the ongoing implementation. The more accurate and careful a
specification is, the more likely that the resulting program will work
well. Even for personal programs, it is valuable to prepare a reasonably
thorough specification because it encourages consideration of alternatives
and records the choices made.

For our purposes, the specification would include function prototypes and a
detailed prescription (命令) of behavior, responsibilities and assumptions:
\begin{myitemize}
\item Fields are separated by commas. \\
    A field may be enclosed in double-quote characters "...". \\
    A quoted field may contain commas but not newlines. \\
    A quoted field may contain double-quote characters ", represented by
    "".   \\
    Fields may be empty; "" and an empty string both represent an empty
    field. \\
    Leading and trailing white space is preserved.
\item \verb'char *csvgetline(FILE *f)';
    \begin{myitemize}
\item reads one line from open input file; assumes that input lines
    are terminated by \verb'\r', \verb'\n', \verb'\r\n', or \verb'EOF'.
\item returns pointer to line, with terminator removed, or \verb'NULL' if
    \verb'EOF' occurred.
\item line may be of arbitrary length; returns \verb'NULL' if memory
    limit exceeded.
\item line must be treated as read-only storage; caller must make a copy
    to preserve or change contents.
\end{myitemize}
\item \verb'char *csvfield(int n)';
\begin{myitemize}
        \item fields are numbered from 0.
        \item returns \verb'n'-th field from last line read by
            \verb'csvgetline'; returns \verb'NULL' if \verb'n < 0' or
            beyond last field.
        \item fields are separated by commas.
        \item fields may be surrounded by \verb'"..."'; such quotes are
            removed; within \verb'"..."', \verb'""' is replaced by \verb'"'
            and comma is not a separator.
        \item in unquoted fields, quotes are regular characters.
        \item there can be an arbitrary number of fields of any length;
            returns \verb'NULL' if memory limit exceeded.
        \item field must be treated as read-only storage; caller must make
            a copy to preserve or change contents.
        \item behavior undefined if called before \verb'csvgetline' is
            called.
\end{myitemize}
\item \verb'int csvnfield(void)';
\begin{myitemize}
        \item returns number of fields on last line read by
        \verb'csvgetlien'.
        \item behavior undefined if called before \verb'csvgetline' is
            called.
\end{myitemize}
\end{myitemize}

This specification still leaves open questions. For example, what values
should be returned by \verb'csvfield' and \verb'csvnfield' if they are
called after \verb'csvgetline' has encountered \verb'EOF'? How should
ill-formed fields be handled? Nailing down (明确)
all such puzzles is difficult
even for a tiny system, and very challenging for a large one, though it is
important to try. One often doesn't discover oversights (失察) and
omissions (遗漏) until implementation is underway (进行中).

The rest of this section contains a new implementation of \verb'csvgetline'
that matches the specification. The library is broken into two files, a
header \verb'csv.h' that contains the function declarations that represent
the public part of the interface, and an implementation file \verb'csv.c'
that contains the code. Users include \verb'csv.h' in their source code and
link their compiled code with the compiled version of \verb'csv.c'; the
source need never be visible.

Here is the header file:
\begin{wellcode}
    /* csv.h: interface for csv library */
    extern char *csvgetline(FILE *f);   /* read next input line */
    extern char *csvfield(int n);       /* return field n */
    extern int  csvnfield(void);        /* return number of fields */
\end{wellcode}

The internal variables that store text and the internal functions like
\verb'split' are declared \verb'static' so they are visible only within the
file that contains them.  This is the simplest way to hide information in a
C program.
\begin{wellcode}
    enum { NOMEM = -2 };            /* out of memory signal */

    static char *line       = NULL; /* input chars */
    static char *sline      = NULL; /* line copy used by split */
    static int  maxline     = 0;    /* size of line[] and sline[] */
    static char **field     = NULL; /* field pointers */
    static int  maxfield    = 0;    /* size of field[] */
    static int  nfield      = 0;    /* number of fields in field[] */

    static char fieldsep[]  = ",";  /* field separator chars */
\end{wellcode}
The variables are initialized statically as well. These initial values are
used to test whether to create or grow arrays.

These declarations describe a simple data structure. The \verb'line' array
holds the input line; the \verb'sline' array is created by copying
characters from \verb'line' and terminating each field. The \verb'field'
array points to entries in \verb'sline'. This diagram shows the state of
these three arrays after the input line \verb'ab,"cd","e""f",,"g,h"' has
been processed. Shaded elements in \verb'sline' are not part of any field.
\\
% TODO: "shaded elements" mentioned in above.
\begin{center}
    \begin{tikzpicture}
        \pgfmathsetmacro\width{0.5};
        \pgfmathsetmacro\height{1.0};
        \pgfmathsetmacro\dist{1.5};

        \node at (0, \height/2) {\texttt{field}};
        \node at (0, \height/2 + \dist) {\texttt{sline}};
        \node at (0, \height/2 + 2*\dist) {\texttt{line}};

        \draw (\dist, \dist) -- (\dist + 21*\width, \dist);
        \draw (\dist, \dist + \height) -- (\dist + 21*\width,
        \dist + \height);
        \draw (\dist, 2*\dist) -- (\dist + 21*\width, 2*\dist);
        \draw (\dist, 2*\dist + \height) -- (\dist + 21*\width,
        2*\dist + \height);

        \newcounter{i};
        \setcounter{i}{0};
        \foreach \ind/\data in { 0/\texttt{a}, 1/\texttt{b},
            2/\texttt{,}, 3/\texttt{"}, 4/\texttt{c},
            5/\texttt{d}, 6/\texttt{"}, 7/\texttt{,}, 8/\texttt{"},
            9/\texttt{e}, 10/\texttt{"}, 11/\texttt{"}, 12/\texttt{f},
            13/\texttt{"}, 14/\texttt{,}, 15/\texttt{,}, 16/\texttt{"},
            17/\texttt{g}, 18/\texttt{,}, 19/\texttt{h}, 20/\texttt{"}
        } {

            \node at (\arabic{i} * \width + \dist + \width / 2, 2 * \dist +
            \height / 2) {\data};
            \draw (\arabic{i} * \width + \dist, 2*\dist) --
            (\arabic{i} * \width + \dist, 2 * \dist + \height);
            \addtocounter{i}{1};
        }
        \draw (\arabic{i} * \width + \dist, 2*\dist) --
        (\arabic{i} * \width + \dist, 2 * \dist + \height);

        \setcounter{i}{0};
        \foreach \ind/\data in { 0/\texttt{a}, 1/\texttt{b},
            2/\texttt{\textbackslash0}, 3/\texttt{"}, 4/\texttt{c},
            5/\texttt{d}, 6/\texttt{\textbackslash0},
            7/\texttt{\textbackslash0}, 8/\texttt{"}, 9/\texttt{e},
            10/\texttt{"}, 11/\texttt{f}, 12/\texttt{\textbackslash0},
            13/\texttt{"}, 14/\texttt{\textbackslash0},
            15/\texttt{\textbackslash0}, 16/\texttt{"}, 17/\texttt{g},
            18/\texttt{,}, 19/\texttt{h}, 20/\texttt{\textbackslash0}
        } {

            \node at (\arabic{i} * \width + \dist + \width / 2, \dist +
            \height / 2) {\data};
            \draw (\arabic{i} * \width + \dist, \dist) --
            (\arabic{i} * \width + \dist, \dist + \height);
            \addtocounter{i}{1};
        }
        \draw (\arabic{i} * \width + \dist, \dist) --
        (\arabic{i} * \width + \dist, \dist + \height);

        \node at (\width / 2 + \dist, \height / 2) {\texttt{0}};
        \node at (4 * \width + \width / 2 + \dist, \height / 2)
        {\texttt{1}};
        \node at (9 * \width + \width / 2 + \dist, \height / 2)
        {\texttt{2}};
        \node at (15 * \width + \width / 2 + \dist, \height / 2)
        {\texttt{3}};
        \node at (17 * \width + \width / 2 + \dist, \height / 2)
        {\texttt{4}};

        \draw[thick, ->] (\width / 2 + \dist, \height / 2 + 0.3) -- (\width
        / 2 + \dist, \dist);

        \draw[thick, ->] (4 * \width + \width / 2 + \dist, \height / 2 +
        0.3) -- (4 * \width + \width / 2 + \dist, \dist);

        \draw[thick, ->] (9 * \width + \width / 2 + \dist, \height / 2 +
        0.3) -- (9 * \width + \width / 2 + \dist, \dist);

        \draw[thick, ->] (15 * \width + \width / 2 + \dist, \height / 2 +
        0.3) -- (15 * \width + \width / 2 + \dist, \dist);

        \draw[thick, ->] (17 * \width + \width / 2 + \dist, \height / 2 +
        0.3) -- (17 * \width + \width / 2 + \dist, \dist);

    \end{tikzpicture}
\end{center}

Here is the function \verb'csvgetline' itself:
\begin{wellcode}
    /* csvgetline: get one line, grow as needed */
    /* sample input: "LU",86,25,"11/4/1998","2:19PM",+4.0625 */
    char *csvgetline(FILE *fin)
    {
        int     i, c;
        char    *newl, *news;

        if (line == NULL) {     /* allocate on first call */
            maxline = maxfield = 1;
            line = (char *)malloc(maxline);
            sline = (char *)malloc(maxline);
            field = (char **)malloc(maxfield * sizeof(field[0]));
            if (line == NULL || sline == NULL || field == NULL) {
                reset();
                return NULL;    /* out of memory */
            }
        }
        for (i = 0; (c=getc(fin)) != EOF && !endofline(fin, c); i++) {
            if (i >= maxline - 1) { /* grow line */
                maxline *= 2;       /* double current size */
                newl = (char *)realloc(line, maxline);
                news = (char *)realloc(sline, maxline);
                if (newl == NULL || news == NULL) {
                    reset();
                    return NULL;    /* out of memory */
                }
                line = newl;
                sline = news;
            }
            line[i] = c;
        }
        line[i] = '\0';
        if (split() == NOMEM) {
            reset();
            return NULL;    /* out of memory */
        }
        return (c == EOF && i == 0) ? NULL : line;
    }
\end{wellcode}
An incoming line is accumulated in \verb'line', which is grown as necessary
by a call to \verb'realloc'; the size is doubled on each growth, as in
Section \ref{sec:growing_arrays}. The \verb'sline' array is kept the same
size as \verb'line'; \verb'csvgetline' calls \verb'split' to create the
field pointers in a separated array \verb'field', which is also grown as
needed.

As is our custom, we start the arrays very small and grow them on demand,
to guarantee that the array-growing code is exercised. If allocation fails,
we call \verb'reset' to restore the globals to their starting state, so a
subsequent call to \verb'csvgetline' has a chance of succeeding:
\begin{wellcode}
    static void reset(void)
    {
        free(line); /* free(NULL) permitted by ANSI C */
        free(sline);
        free(field);
        line = NULL;
        sline = NULL;
        field = NULL;
        maxline = maxfield = nfield = 0;
    }
\end{wellcode}

The \verb'endofline' function handles the problem that an input line may be
terminated by a carriage return, a newline, both, or even \verb'EOF':
\begin{wellcode}
    /* endofline: check for and consume \r, \n, \r\n, or EOF */
    static int endofline(FILE *fin, int c)
    {
        int eol;

        eol = (c == '\r' || c == '\n');
        if (c == '\r') {
            c = getc(fin);
            if (c != '\n' && c != EOF)
                ungetc(c, fin); /* read too far; put c back */
        }
        return eol;
    }
\end{wellcode}
A separate function is necessary, since the standard input functions do not
handle the rich variety of perverse formats encountered in real inputs.

Our prototype used \verb'strtok' to find the next token by searching for a
separator character, normally a comma, but this made it impossible to
handle quoted commas.  A major change in the implementation of \verb'split'
is necessary, though its interface need not change. Consider these input
lines:
\begin{wellcode}
    "",,""
    ,"",
    ,,
\end{wellcode}

Each line has three empty fields. Making sure that \verb'split' parses them
and other odd (古怪的) inputs correctly complicates it significantly, an
example of how special cases and boundary conditions can come to dominate a
program.
\begin{wellcode}
    /* split: split line into fields */
    static int split(void)
    {
        char    *p, **newf;
        char    *sepp;  /* pointer to temporary separator character */
        int     sepc;   /* temporary separator character */

        nfield = 0;
        if (line[0] == '\0')
            return 0;
        strcpy(sline, line);
        p = sline;

        do {
            if (nfield >= maxfield) {
                maxfield *= 2;  /* double current size */
                newf = (char **)realloc(field,
                        maxfield * sizeof(field[0]));
                if (newf == NULL)
                    return NOMEM;
                field = newf;
            }
            if (*p == '"')
                sepp = advquoted(++p);  /* skip initial quote */
            else
                sepp = p + strcspn(p, fieldsep);
            sepc = sepp[0];
            sepp[0] = '\0'; /* terminate field */
            field[nfield++] = p;
            p = sepp + 1;
        } while (sepc == ',');

        return nfield;
    }
\end{wellcode}
The loop grows the array of field pointers if necessary, then calls one of
two other functions to locate and process the next field. If the field
begins with a quote, \verb'advquoted' finds the field and returns a pointer
to the separator that ends the field.  Otherwise, to find the next comma we
use the library function \verb'strcspn(p, s)', which searches a string
\verb'p' for the next occurrence of any character in string \verb's'; it
returns the number of characters skipped over.

Quotes within a field are represented by two adjacent quotes, so
\verb'advquoted' squeezes those into a single one; it also removes the
quotes that surround the field.  Some complexity is added by an attempt to
cope with plausible (貌似合理的) inputs that don't match the specification,
such as \verb'"abc"def'. In such cases, we append whatever follows the
second quote until the next separator as part of this field.  Microsoft
Excel appears to use a similar algorithm.
\begin{wellcode}
    /* advquoted: quoted field; return pointer to next separator */
    static char *advquoted(char *p)
    {
        int i, j;

        for (i = j = 0; p[j] != '\0'; i++, j++) {
            if (p[j] == '"' && p[++j] != '"') {
                /* copy up to next separator or \0 */
                int k = strcspn(p+j, fieldsep);
                memmove(p+i, p+j, k);
                i += k;
                j += k;
                break;
            }
            p[i] = p[j];
        }
        p[i] = '\0';
        return p + j;
    }
\end{wellcode}

Since the input line is already split, \verb'csvfield' and \verb'csvnfield'
are trivial (琐细的).
\begin{wellcode}
    /* csvfield: return pointer to n-th field */
    char *csvfield(int n)
    {
        if (n < 0 || n >= nfield)
            return NULL;
        return field[n];
    }

    /* csvnfield: return number of fields */
    int csvnfield(void)
    {
        return nfield;
    }
\end{wellcode}

Finally, we can modify the test driver to exercise this version of the
library; since it keeps a copy of the input line, which the prototype does
not, it can print the original line before printing the fields:
\begin{wellcode}
    /* csvtest main: test CSV library */
    int main(void)
    {
        int     i;
        char    *line;

        while ((line = csvgetline(stdin)) != NULL) {
            printf("line = '%s'\n", line);
            for (i = 0; i < csvnfield(); i++)
                printf("field[%d] = '%s'\n", i, csvfield(i));
        }
        return 0;
    }
\end{wellcode}

This completes our C version. It handles arbitrarily large inputs and does
something sensible even with perverse (歪曲的) data. The price is that it
is more than four times as long as the first prototype and some of the code
is intricate (复杂的). Such expansion of size and complexity is a typical
result of moving from prototype to production.

\begin{exercise}
    There are several degrees of laziness for field-splitting; among the
    possibilities are to split all at once but only when some field is
    requested, to split only the field requested, or to split up to the
    field requested. Enumerate possibilities, assess their potential
    difficulty and benefits, then write them and measure their speeds.
\end{exercise}

\begin{exercise}
    Add a facility so separators can be changed (a) to an arbitrary class
    of characters; (b) to different separators for different fields; (c) to
    a regular expression (see Chapter \ref{chap:notation}). What should the
    interface look like?
\end{exercise}

\begin{exercise}
    We chose to use the static initialization provided by C as the basis of
    a one-time switch: if a pointer is \verb'NULL' on entry, initialization
    is performed. Another possibility is to require the user to call an
    explicit initialization function, which could include suggested initial
    sizes for arrays. Implement a version that combines the best of both.
    What is the role of \verb'reset' in your implementation ?
\end{exercise}

\begin{exercise}
    Design and implement a library for creating CSV-formatted data. The
    simplest version might take an array of strings and print them with
    quotes and commas. A more sophisticated version might use a format
    string analogous to \verb'printf'.  Look at Chapter \ref{chap:notation}
    for some suggestions on notation.
\end{exercise}

\section{A C++ Implementation}
\label{sec:a_c++_implementation}

In this section we will write a C++ version of the CSV library to address
some of the remaining limitations of the C version. This will entail (需要)
some changes to the specification, of which the most important is that the
functions will handle C++ strings instead of C character arrays. The use of
C++ strings will automatically resolve some of the storage management
issues, since the library functions will manage the memory for us. In
particular, the field routines will return strings that can be modified by
the caller, a more flexible design than the previous version.

A class \verb'Csv' defines the public face, while neatly hiding the
variables and functions of the implementation. Since a class object
contains all the state for an instance, we can instantiate multiple
\verb'Csv' variables; each is independent of the others so multiple CSV
input streams can operate at the same time.
\begin{wellcode}
    class Csv { // read and parse Comma-separated values
        // sample input: "LU",86.25,"11/4/1998","2:19PM",+4.0625
        public:
            Csv(istream& fin = cin, string sep = ","):
                fin(fin), fieldsep(sep) {}

            int getline(string&);
            string getfield(int n);
            int getnfield() const { return nfield; }

        private:
            istream& fin;           // input file pointer
            string line;            // input line
            vector<string> field;   // field strings
            int nfield;             // number of fields
            string fieldsep;        // separator characters

            int split();
            int endofline(char);
            int advplain(const string& line, string& fld, int);
            int advquoted(const string& line, string& fld, int);
    };
\end{wellcode}
Default parameters for the constructor are defined so a default \verb'Csv'
object will read from the standard input stream and use the normal field
separator; either can be replaced with explicit values.

To manage strings, the class uses the standard C++ \verb'string' and
\verb'vector' classes rather than C-style strings. There is no non-existent
state for a \verb'string': "empty" means only that the length is zero, and
there is no equivalent of \verb'NULL', so we can't use that as an end of
file signal. Thus \verb'Csv::getline' returns the input line through an
argument by reference, reserving the function value itself for end of file
and error reports.
\begin{wellcode}
    // getline: get one line, grow as needed
    int Csv::getline(string& str)
    {
        char    c;

        for (line = ""; fin.get(c) && !endofline(c); )
            line += c;
        split();
        str = line;
        return !fin.eof();
    }
\end{wellcode}
The \verb'+=' operator is overloaded to append a character to a string.

Minor changes are needed in \verb'endofline'. Again, we have to read the
input a character at a time, since none of the standard input routines can
handle the variety of inputs.
\begin{wellcode}
    // endofline: check for and consume \r, \n, \r\n, or EOF
    int Csv::endofline(char c)
    {
        int eol;

        eol = (c == '\r' || c == '\n');
        if (c == '\r') {
            fin.get(c);
            if (!fin.eof() && c != '\n')
                fin.putback(c); // read too far
        }
        return eol;
    }
\end{wellcode}
Here is the new version of \verb'split':
\begin{wellcode}
    // split: split line into fields
    int Csv::split()
    {
        string  fld;
        int     i, j;

        nfield = 0;
        if (line.length() == 0)
            return 0;
        i = 0;

        do {
            if (i < line.length() && line[i] == '"')
                j = advquoted(line, fld, ++i);  // skip quote
            else
                j = advplain(line, fld, i);
            if (nfield >= field.size())
                field.push_back(fld);
            else
                field[nfield] = fld;
            nfield++;
            i = j + 1;
        } while (j < line.length());
        return nfield;
    }
\end{wellcode}

Since \verb'strcspn' doesn't work on C++ strings, we must change both
\verb'split' and \verb'advquoted'. The new version of \verb'advquoted' uses
the C++ standard function \verb'find_first_of' to locate the next
occurrence of a separator character. The call
\verb's.find_first_of(fieldsep, j)' searches the string \verb's' for the
first instance of any character in \verb'fieldsep' that occurs at or after
position \verb'j'. If it fails to find an instance, it returns an index
beyond the end of the string, so we must bring it back within range.  The
inner \verb'for' loop that follows appends characters up to the separator
to the field being accumulated in \verb'fld'.
\begin{wellcode}
    // advquoted: quoted field; return index of next separator
    int Csv::advquoted(const string& s, string& fld, int i)
    {
        int j;

        fld = "";
        for (j = i; j < s.length(); j++) {
            if (s[j] == '"' && s[++j] != '"') {
                int k = s.find_first_of(fieldsep, j);
                if (k > s.length()) // no separator found
                    k = s.length();
                for (k -= j; k-- > 0; )
                    fld += s[j++];
                break;
            }
            fld += s[j];
        }
        return j;
    }
\end{wellcode}

The function \verb'find_first_of' is also used in a new function
\verb'advplain', which advances over a plain unquoted field. Again, this
change is required because C string functions like \verb'strcspn' cannot be
applied to C++ strings, which are an entirely different data type.
\begin{wellcode}
    // advplain: unquoted field; return index of next separator
    int Csv::advplain(const string& s, string& fld, int i)
    {
        int j;

        j = s.find_first_of(fieldsep, i);   // book for separator
        if (j > s.length())
            j = s.length();
        fld = string(s, i, j-i);
        return j;
    }
\end{wellcode}

As before, \verb'Csv::getfield' is trivial, while \verb'Csv::getnfield' is
so short that it is implemented in the class definition.
\begin{wellcode}
    // getfield: return n-th field
    string Csv::getfield(int n)
    {
        if (n < 0 || n >= nfield)
            return "";
        else
            return field[n];
    }
\end{wellcode}

Our test program is a simple variant of the earlier one:
\begin{wellcode}
    // Csvtest main: test Csv class
    int main(void)
    {
        string  line;
        Csv     csv;

        while (csv.getline(line) != 0) {
            cout << "line = '" << line << "'\n";
            for (int i = 0; i < csv.getnfield(); i++)
                cout << "field[" << i << "] = '"
                    << csv.getfield(i) << "'\n";
        }
        return 0;
    }
\end{wellcode}

The usage is different than with the C version, though only in a minor way.
Depending on the compiler, the C++ version is anywhere from 40 percent to
four times slower than the C version on a large input file of 30,000 lines
with about 25 fields per line. As we saw when comparing versions of
\verb'markov', this variability is a reflection on library maturity
(成熟度). The C++ source program is about 20 percent shorter.

\begin{exercise}
    Enhance the C++ implementation to overload subscripting with operator
    \verb'[]' so that fields can be accessed as \verb'csv[i]'.
\end{exercise}

\begin{exercise}
    Write a Java version of the CSV library, then compare the three
    implementations for clarity, robustness, and speed.
\end{exercise}

\begin{exercise}
    Repackage the C++ version of the CSV code as an STL iterator.
\end{exercise}

\begin{exercise}
    \label{exer:csvnew}
    The C++ version permits multiple independent \verb'Csv' instances to
    operate concurrently without interfering, a benefit of encapsulating
    all the state in an object that can be instantiated multiple times.
    Modify the C version to achieve the same effect by replacing the global
    data structures with structures that are allocated and initialized by
    an explicit \verb'csvnew' function.
\end{exercise}

\section{Interface Principles}
\label{sec:interface_principles}

In the previous sections we were working out the details of an interface,
which is the detailed boundary between code that provides a service and
code that uses it. An interface defines what some body of code does for its
users, how the functions and perhaps data members can be used by the rest
of the program. Our CSV interface provides three functions -- read a line,
get a field, and return the number of fields -- which are the only
operations that can be performed.

To prosper (成功), an interface must be well suited (使适应) for its task
-- simple, general,  regular, predictable, robust -- and it must adapt
gracefully as its users and its implementation change. Good interfaces
follow a set of principles. These are not independent or even consistent,
but they help us describe what happens across the boundary between two
pieces of software.

\emph{Hide implementation details.} The implementation behind the interface
should be hidden from the rest of the program so it can be changed without
affecting or breaking anything. There are several terms for this kind of
organizing principle; information hiding, encapsulation, abstraction,
modularization, and the like all refer to related ideas. An interface
should hide details of the implementation that are irrelevant to the client
(user) of the interface. Details that are invisible can be changed without
affecting the client, perhaps to extend the interface, make it more
efficient, or even replace its implementation altogether.

The basic libraries of most programming languages provide familiar
examples, though not always especially well-designed ones. The C standard
I/O library is among the best known: a couple of dozen functions that open,
close, read, write, and otherwise manipulate files. The implementation of
file I/O is hidden behind a data type \verb'FILE *', whose properties one
might be able to see (because they are often spelled out in
\verb'<stdio.h>') but should not exploit (利用).

If the header file does not include the actual structure declaration, just
the name of the structure, this is sometimes called an \term{opaque type},
since its properties are not visible and all operations take place through
a pointer to whatever real object lurks (隐藏) behind.

Avoid global variables; wherever possible it is better to pass references
to all data through function arguments.

We strongly recommend against publicly visible data in all forms; it is too
hard to maintain consistency of values if users can change variables at
will.  Function interfaces make it easier to enforce access rules, but this
principle is often violated. The predefined I/O streams like \verb'stdin'
and \verb'stdout' are almost always defined as elements of a global array
of \verb'FILE' structures:
\begin{wellcode}
    extern  FILE    __iob[_NFILE];
    #define stdin   (&__iob[0])
    #define stdout  (&__iob[1])
    #define stderr  (&__iob[2])
\end{wellcode}

This makes the implementation completely visible; it also means that one
can't assign to \verb'stdin', \verb'stdout' or \verb'stderr', even though
they look like variables.  The peculiar (罕见的) name \verb'__iob' uses the
ANSI C convention of two leading underscores for private names that must be
visible, which makes the names less likely to conflict with names in a
program.

Classes in C++ and Java are better mechanisms for hiding information; they
are central to the proper use of those languages. The container classes of
the C++ Standard Template Library that we used in Chapter \ref{chap:desipl}
carry this even further: aside from some performance guarantees there is no
information about implementation, and library creators can use any
mechanism they like.

\emph{Choose a small orthogonal (正交的) set of primitives.} An interface
should provide as much functionality as necessary but no more, and the
functions should not overlap excessively (过度地) in their capabilities.
Having lots of functions may make the library easier to use -- whatever one
needs is there for the taking. But a large interface is harder to write and
maintain, and sheer (陡峭的) size may make it hard to learn and use as
well. "Application program interfaces" or APIs are sometimes so huge that
no mortal (人) can be expected to master them.

In the interest of convenience, some interfaces provide multiple ways of
doing the same thing, a tendency that should be resisted (反抗). The C
standard I/O library provides at least four different functions that will
write a single character to an output stream:
\begin{wellcode}
    char    c;
    putc(c, fp);
    fputc(c, fp);
    fprintf(fp, "%c", c);
    fwrite(&c, sizeof(char), 1, fp);
\end{wellcode}
If the stream is \verb'stdout', there are several more possibilities. These
are convenient, but not all are necessary.

Narrow interfaces are to be preferred to wide ones, at least until one has
strong evidence that more functions are needed. Do one thing, and do it
well.  Don't add to an interface just because it's possible to do so, and
don't fix the interface when it's the implementation that's broken. For
instance, rather than having \verb'memcpy' for speed and \verb'memmove' for
safety, it would be better to have one function that was always safe, and
fast when it could be.

\emph{Don't reach behind the user's back.} A library function should not
write secret files and variables or change global data, and it should be
circumspect (慎重的) about modifying data in its caller. The \verb'strtok'
function fails several of these criteria. It is a bit of a surprise that
\verb'strtok' writes null bytes into the middle of its input string. Its
use of the null pointer as a signal to pick up where it left off last time
implies secret data held between calls, a likely source of bugs, and it
precludes (排除) concurrent uses of the function. A better design would
provide a single function that tokenizes an input string.  For similar
reasons, our second C version can't be used for two input streams; see
Exercise \ref{exer:csvnew}.

The use of one interface should not demand (要求) another one just for the
convenience of the interface designer or implementer. Instead, make the
interface self-contained, or failing that, be explicit about what external
services are required.  Otherwise, you place a maintenance burden on the
client. An obvious example is the pain of managing huge lists of header
files in C and C++ source; header files can be thousands of lines long and
include dozens of other headers.

\emph{Do the same thing the same way everywhere.} Consistency and
regularity are important. Related things should be achieved by related
means. The basic \verb'str...' functions in the C library are easy to use
without documentation because they all behave about the same: data flows
from right to left, the same direction as in an assignment statement, and
they all return the resulting string. On the other hand, in the C Standard
I/O library it is hard to predict the order of arguments to functions.
Some have the \verb'FILE *' argument first, some last; others have various
orders for size and number of elements. The algorithms for STL containers
present a very uniform interface, so it is easy to predict how to use an
unfamiliar function.

External consistency, behaving like something else, is also a goal. For
example, the \verb'mem...' functions were designed after the \verb'str...'
functions in C, but borrowed their style. The standard I/O functions
\verb'fread' and \verb'fwrite' would be easier to remember if they looked
like the \verb'read' and \verb'write' functions they were based on. Unix
command-line options are introduced by a minus sign, but a given option
letter may mean completely different things, even between related programs.

If wildcards like the \verb'*' in \verb'*.exe' are all expanded by a
command interpreter, behavior is uniform. If they are expanded by
individual programs, non-uniform behavior is likely. Web browsers take a
single mouse click to follow a link, but other applications take two clicks
to start a program or follow a link; the result is that many people
automatically click twice regardless.

These principles are easier to follow in some environments than others, but
they still stand (停滞). For instance, it's hard to hide implementation
details in C, but a good programmer will not exploit them, because to do so
makes the details part of the interface and violates the principle of
information hiding. Comments in header files, names with special forms
(such as \verb'__iob'), and so on are ways of encouraging good behavior
when it can't be enforced (执行).

No matter what, there is a limit to how well we can do in designing an
interface.  Even the best interfaces of today may eventually become the
problems of tomorrow, but good design can push tomorrow off a while longer.

\section{Resource Management}
\label{sec:resource_management}

One of the most difficult problems in designing the interface for a library
(or a class or a package) is to manage resources that are owned by the
library or that are shared by the library and those who call it. The most
obvious such resource is memory -- who is responsible for allocating and
freeing storage? -- but other shared resources include open files and the
state of variables whose values are of common interest (共同关心的).
Roughly, the issues fall into the categories of initialization, maintaining
state, sharing and copying, and cleaning up.

The prototype of our CSV package used static initialization to set the
initial values for pointers, counts, and the like. But this choice is
limiting since it prevents restarting the routines in their initial state
once one of the functions has been called. An alternative is to provide an
initialization function that sets all internal values to the correct
initial values. This permits restarting, but relies on the user to call it
explicitly. The \verb'reset' function in the second version could be made
public for this purpose.

In C++ and Java, constructors are used to initialize data members of
classes.  Properly defined constructors ensure that all data members are
initialized and that there is no way to create an uninitialized class
object. A group of constructors can support various kinds of initializers;
we might provide \verb'Csv' with one constructor that takes a file name and
another that takes an input stream.

What about copies of information managed by a library, such as the input
lines
and fields? Our C \verb'csvgetline' program provides direct access to the input
strings
(line and fields) by returning pointers to them. This unrestricted access
has several
drawbacks. It's possible for the user to overwrite memory so as to render
other information invalid; for example, an expression like
\begin{wellcode}
    strcpy(csvfield(1), csvfield(2));
\end{wellcode}
could fail in a variety of ways, most likely by overwriting the beginning
of field 2 if field 2 is longer than field 1. The user of the library must
make a copy of any information to be preserved beyond the next call to
\verb'csvgetline'; in the following sequence, the pointer might well be
invalid at the end if the second \verb'csvgetline' causes a reallocation of
its line buffer.
\begin{wellcode}
    char    *p;

    csvgetline(fin);
    p = csvfield(1);
    csvgetline(fin);
    /* p could be invalid here */
\end{wellcode}
The C++ version is safer because the strings are copies that can be changed
at will.

Java uses references to refer to objects, that is, any entity other than
one of the basic types like \verb'int'. This is more efficient than making
a copy, but one can be fooled into thinking that a reference is a copy; we
had a bug like that in an early version of our Java \verb'markov' program
and this issue is a perennial (常见的) source of bugs involving strings in
C. Clone methods provide a way to make a copy when necessary.

The other side of initialization or construction is finalization or
destruction -- cleaning up and recovering resources when some entity is no
longer needed.  This is particularly important for memory, since a program
that fails to recover unused memory will eventually run out. Much modern
software is embarrassingly prone (倾向于) to this fault. Related problems
occur when open files are to be closed: if data is being buffered, the
buffer may have to be flushed (and its memory reclaimed). For standard C
library functions, flushing happens automatically when the program
terminates normally, but it must otherwise be programmed
(必须再提供一种方式). The C and C++ standard function \verb'atexit'
provides a way to get control just before a program terminates normally;
interface implementers can use this facility to schedule cleanup.

\emph{Free a resource in the same layer that allocated it.} One way to
control resource allocation and reclamation is to have the same library,
package, or interface that allocates a resource be responsible for freeing
it. Another way of saying this is that the allocation state of a resource
should not change across the interface. Our CSV libraries read data from
files that have already been opened, so they leave them open when they are
done. The caller of the library needs to close the files.

C++ constructors and destructors help enforce this rule. When a class
instance goes out of scope or is explicitly destroyed, the destructor is
called; it can flush buffers, recover memory, reset values, and do whatever
else is necessary.  Java does not provide an equivalent mechanism. Although
it is possible to define a finalization method for a class, there is no
assurance that it will run at all, let alone at a particular time, so
cleanup actions cannot be guaranteed to occur, although it is often
reasonable to assume they will.

Java does provide considerable help with memory management because it has
built-in \term{garbage collection}. As a program runs, it allocates new
objects.  There is no way to deallocate them explicitly, but the run-time
system keeps track of which objects are still in use and which are not, and
periodically returns unused ones to the available memory pool.

There are a variety of techniques for garbage collection. Some schemes keep
track of the number of uses of each object, its \term{reference count}, and
free an object when its reference count goes to zero. This technique can be
used explicitly in C and C++ to manage shared objects. Other algorithms
periodically follow a trail (踪迹) from the allocation pool to all
referenced objects. Objects that are found this way are still in use;
objects that are not referred to by any other object are not in use and can
be reclaimed.

The existence of automatic garbage collection does not mean that there are
no memory-management issues in a design. We still have to determine whether
interfaces return references to shared objects or copies of them, and this
affects the entire program. Nor is garbage collection free -- there is
overhead to maintain information and to reclaim unused memory, and
collection may happen at unpredictable times.

All of these problems become more complicated if a library is to be used in
an environment where more than one thread of control can be executing its
routines at the same time, as in a multi-threaded Java program.

To avoid problems, it is necessary to write code that is \term{reentrant},
which means that it works regardless of the number of simultaneous
executions.  Reentrant code will avoid global variables, static local
variables, and any other variable that could be modified while another
thread is using it. The key to good multi-thread design is to separate the
components so they share nothing except through well-defined interfaces.
Libraries that inadvertently (不慎地) expose variables to sharing destroy
the model.  (In a multi-thread program, \verb'strtok' is a disaster, as are
other functions in the C library that store values in internal static
memory.) If variables might be shared, they must be protected by some kind
of locking mechanism to ensure that only one thread at a time accesses
them. Classes are a big help here because they provide a focus for
discussing sharing and locking models. Synchronized methods in Java provide
a way for one thread to lock an entire class or instance of a class against
simultaneous modification by some other thread; synchronized blocks permit
only one thread at a time to execute a section of code.

Multi-threading adds significant complexity to programming issues, and is
too big a topic for us to discuss in detail here.

\section{Abort, Retry, Fail?}
\label{sec:abort_retry_fail}

In the previous chapters we used functions like \verb'eprintf' and
\verb'estrdup' to handle errors by displaying a message before terminating
execution. For example, \verb'eprintf' behaves like
\verb'fprintf(stderr, ...)', but exits the program with an error status
after reporting the
error. It uses the \verb'<stdarg.h>' header and the \verb'vfprintf'
library routine to print the arguments represented by the \verb'...' in the
prototype. The \verb'stdarg' library must be initialized by a call to
\verb'va_start' and terminated by \verb'va_end'.  We will use more of this
interface in Chapter \ref{chap:notation}.
\begin{wellcode}
    #include <stdarg.h>
    #include <string.h>
    #include <errno.h>

    /* eprintf: print error message and exit */
    void eprintf(char *fmt, ...)
    {
        va_list args;

        fflush(stdout);
        if (progname() != NULL)
            fprintf(stderr, "%s: ", progname());

        va_start(args, fmt);
        vfprintf(stderr, fmt,args);
        va_end(args);

        if (fmt[0] != '\0' && fmt[strlen(fmt)-1] == ':')
            fprintf(stderr, " %s", strerror(errno));
        fprintf(stderr, "\n");
        exit(2);    /* conventional value for failed execution */
    }
\end{wellcode}
If the format argument ends with a colon, \verb'eprintf' calls the standard
C function \verb'strerror', which returns a string containing any
additional system error information that might be available. We also wrote
\verb'weprintf', similar to \verb'eprintf', that displays a warning but
does not exit. The \verb'printf'-like interface is convenient for building
up strings that might be printed or displayed in a dialog box.

Similarly, \verb'estrdup' tries to make a copy of a string, and exits with
a message (via \verb'eprintf') if it runs out of memory:
\begin{wellcode}
    /* estrdup: duplicate a string, report if error */
    char *estrdup(char *s)
    {
        char    *t;

        t = (char *)malloc(strlen(s)+1);
        if (t == NULL)
            eprintf("estrdup(\"%.20s\") failed: ", s);
        strcpy(t, s);
        return t;
    }
\end{wellcode}
and \verb'emalloc' provides a similar service for calls to \verb'malloc':
\begin{wellcode}
    /* emalloc: malloc and report if error */
    void *emalloc(size_t n)
    {
        void *p;

        p = malloc(n);
        if (p == NULL)
            eprintf("malloc of %u bytes failed: ", n);
        return p;
    }
\end{wellcode}

A matching header file called \verb'eprintf.h' declares these functions:
\begin{wellcode}
    /* eprintf.h: error wrapper functions */
    extern void eprintf(char *, ...);
    extern void weprintf(char *, ...);
    extern char *estrdup(char *);
    extern void *emalloc(size_t);
    extern void *nerealloc(void *, size_t);
    extern char *progname(void);
    extern void setprogname(char *);
\end{wellcode}
This header is included in any file that calls one of the error functions.
Each error message also includes the name of the program if it has been set
by the caller: this is set and retrieved by the trivial (琐碎的) functions
\verb'setprogname' and \verb'progname', declared in the header file and
defined in the source file with \verb'eprintf':
\begin{wellcode}
    static char *name = NULL; /* program name for message */

    /* setprogname: set stored name of program */
    void setprogname(char *str)
    {
        name = estrdup(str);
    }

    /* progname: return stored name of program */
    char *progname(void)
    {
        return name;
    }
\end{wellcode}
Typical usage looks like this:
\begin{wellcode}
    int main(int argc, char *argv[])
    {
        setprogname("markov");
        ...
        f = fopen(argv[i], "r");
        if (f == NULL)
            eprintf("can't open %s:", argv[i]);
        ...
    }
\end{wellcode}
which prints output like this:
\begin{wellcode}
    markov: can't open psalm.txt: No such file or directory.
\end{wellcode}

We find these wrapper functions convenient for our own programming, since
they unify error handling and their very (特有的)
existence encourages us to catch
errors instead of ignoring them. There is nothing special about our design,
however, and you might prefer some variant for your own programs.

Suppose that rather than writing functions for our own use, we are creating
a library for others to use in their programs. What should a function in
that library do if an unrecoverable error occurs? The functions we wrote
earlier in this chapter display a message and die. This is acceptable
behavior for many programs, especially small stand-alone tools and
applications. For other programs, however, quitting is wrong since it
prevents the rest of the program from attempting any recovery; for
instance, a word processor must recover from errors so it does not lose the
document that you are typing. In some situations a library routine should
not even display a message, since the program may be running in an
environment where a message will interfere with displayed data or disappear
without a trace. A useful alternative is to record diagnostic output in an
explicit "log file", where it can be monitored independently.

\emph{Detect errors at a low level, handle them at a high level.} As a
general principle, errors should be detected at as low a level as possible,
but handled at a high level. In most cases, the caller should determine how
to handle an error, not the callee. Library routines can help in this by
failing gracefully; that reasoning (论证)
led us to return \verb'NULL' for a
non-existent field rather than aborting. Similarly, \verb'csvgetline'
returns \verb'NULL' no matter how many times it is called after the first
end of file.

Appropriate return values are not always obvious, as we saw in the earlier
discussion about what \verb'csvgetline' should return. We want to return as
much useful information as possible, but in a form that is easy for the
rest of the program to use. In C, C++ and Java, that means returning
something as the function value, and perhaps other values through reference
(pointer) arguments. Many library functions rely on the ability to
distinguish normal values from error values. Input functions like
\verb'getchar' return a \verb'char' for valid data, and some
non-\verb'char' value like \verb'EOF' for end of file or error.

This mechanism doesn't work if the function's legal return values take up
all possible values. For example a mathematical function like \verb'log'
can return any floating-point number. In IEEE floating point, a special
value called \verb'NaN' ("not a number") indicates an error and can be
returned as an error signal.

Some languages, such as Perl and Tcl, provide a low-cost way to group two
or more values into a \term{tuple}. In such languages, a function value and
any error state can be easily returned together. The C++ STL provides a
pair data type that can also be used in this way.

It is desirable to distinguish various exceptional values like end of file
and error states if possible, rather than lumping them together
(把...放在一起) into a single value.  If the values can't readily (轻易地)
be separated, another option is to return a single "exception" value and
provide another function that returns more detail about the last error.

This is the approach used in Unix and in the C standard library, where many
system calls and library functions return \verb'-1' but also set a global
variable called \verb'errno' that encodes the specific error;
\verb'strerror' returns a string associated with the error number. On our
system, this program:
\begin{wellcode}
    #include <stdio.h>
    #include <string.h>
    #include <errno.h>
    #include <math.h>

    /* errno main: test errno */
    int main(void)
    {
        double  f;

        errno = 0;  /* clear error state */
        f = log(-1.23);
        printf("%f %d %s\n", f, errno, strerror(errno));
        return 0;
    }
\end{wellcode}
prints
\begin{wellcode}
    nan0x10000000 33 Domain error
\end{wellcode}
As shown, \verb'errno' must be cleared first; then if an error occurs,
\verb'errno' will be set to a non-zero value.

\emph{Use exceptions only for exceptional situations.} Some languages
provide exceptions to catch unusual situations and recover from them; they
provide an alternate flow of control when something bad happens. Exceptions
should not be used for handling expected return values. Reading from a file
will eventually produce an end of file; this should be handled with a
return value, not by an exception.

In Java, one writes
\begin{wellcode}
    String fname = "someFileName";
    try {
        FileInputStream in = new FileInputStream(fname);
        int c;
        while ((c = in.read()) != -1)
            System.out.print((char)c);
        in.close();
    } catch (FileNotFoundException e) {
        System.err.println(fname, " not found");
    } catch (IOException e) {
        System.err.println("IOException: " + e);
        e.printStackTrace();
    }
\end{wellcode}

The loop reads characters until end of file, an expected event that is
signaled by a return value of \verb'-1' from \verb'read'.
If the file can't be
opened, that raises an exception, however, rather than setting the input
stream to \verb'null' as would be done in C or C++.  Finally, if some other
I/O error happens in the \verb'try' block, it is also exceptional, and it
is caught by the \verb'IOException' clause (子句).

Exceptions are often overused. Because they distort (扭曲) the flow of
control, they can lead to convoluted (复杂的) constructions that are prone
to bugs. It is hardly exceptional to fail to open a file; generating an
exception in this case strikes us as over-engineering.  Exceptions are best
reserved for truly unexpected events, such as file systems filling up or
floating-point errors.

For C programs, the pair of functions \verb'setjmp' and \verb'longjmp'
provide a much lower-level service upon which an exception mechanism can be
built, but they are sufficiently arcane (神秘的) that we won't go into them
here.

What about recovery of resources when an error occurs? Should a library
attempt a recovery when something goes wrong? Not usually, but it might do
a service by making sure that it leaves information in as clean and
harmless a state as possible.  Certainly unused storage should be
reclaimed. If variables might be still accessible, they should be set to
sensible values. A common source of bugs is trying to use a pointer that
points to freed storage. If error-handling code sets pointers to zero after
freeing what they point to, this won't go undetected. The \verb'reset'
function in the second version of the CSV library was an attempt to address
these issues. In general, aim to keep the library usable after an error has
occurred.

\section{User Interfaces}
\label{sec:user_interfaces}

Thus far we have talked mainly about interfaces among the components of a
program or between programs. But there is another important kind of
interface, between a program and its human users.

Most of the example programs in this book are text-based, so their user
interfaces tend to be straightforward. As we discussed in the previous
section, errors should be detected and reported, and recovery attempted
where it makes sense. Error output should include all available information
and should be as meaningful as possible out of context; a diagnostic should
not say
\begin{wellcode}
    estrdup failed
\end{wellcode}
when it could say
\begin{wellcode}
    markov: estrdup("Derrida") failed: Memory limit reached
\end{wellcode}
It costs nothing to add the extra information as we did in \verb'estrdup',
and it may help a user to identify a problem or provide valid input.

Programs should display information about proper usage when an error is
made, as shown in functions like
\begin{wellcode}
    /* usage: print usage message and exit */
    void usage(void)
    {
        fprintf(stderr, "usage: %s [-d] [-n nwords]",
            " [-s seed] [files ...]\n", progname());
        exit(2);
    }
\end{wellcode}

The program name identifies the source of the message, which is especially
important if this is part of a larger process. If a program presents a
message that just says syntax error or \verb'estrdup' failed , the user
might have no idea who said it.

The text of error messages, prompts, and dialog boxes should state (陈述)
the form
of valid input. Don't say that a parameter is too large; report the valid
range of values.  When possible, the text should be valid input itself,
such as the full command line with the parameter set properly. In addition
to steering (指示) users toward proper use, such output can be captured in
a file or by a mouse sweep and then used to run some further process. This
points out a weakness of dialog boxes: their contents are hard to grab for
later use.

One effective way to create a good user interface for input is by designing
a specialized language for setting parameters, controlling actions, and so
on; a good notation can make a program easy to use while it helps organize
an implementation.  Language-based interfaces are the subject of Chapter
\ref{chap:notation}.

Defensive programming, that is, making sure that a program is invulnerable
to bad input, is important both for protecting users against themselves and
also as a security mechanism. This is discussed more in Chapter
\ref{chap:testing}, which talks about program testing.

For most people, graphical interfaces are the user interface for their
computers.  Graphical user interfaces are a huge topic, so we will say only
a few things that are germane (有关的) to this book. First, graphical
interfaces are hard to create and make "right" since their suitability and
success depend strongly on human behavior and expectations. Second, as a
practical matter, if a system has a user interface, there is usually more
code to handle user interaction than there is in whatever algorithms do the
work.

Nevertheless, familiar principles apply to both the external design and the
internal implementation of user interface software. From the user's
standpoint (观点), style issues like simplicity, clarity, regularity,
uniformity, familiarity, and restraint (约束) all contribute to an
interface that is easy to use; the absence of such properties usually goes
along with unpleasant or awkward interfaces.

Uniformity and regularity are desirable (令人渴望的), including consistent
use of terms, units, formats, layouts, fonts, colors, sizes, and all the
other options that a graphical system makes available. How many different
English words are used to exit from a program or close a window? The
choices range from Abandon to \verb'Control-Z', with at least a dozen
between. This inconsistency is confusing to a native speaker and baffling
(困惑的) for others.

Within graphics code, interfaces are particularly important, since these
systems are large, complicated, and driven by a very different input model
than scanning sequential text. Object-oriented programming excels (擅长) at
graphical user interfaces, since it provides a way to encapsulate all the
state and behaviors of windows, using inheritance to combine similarities
in base classes while separating differences in derived classes.

\section*{Supplementary Reading}

Although a few of its technical details are now dated. \bookname{The
Mythical Man-Month}, by Frederick P. Brooks, Jr. (Addison-Wesley, 1975;
Anniversary Edition 1995), is delightful reading and contains insights
about software development that are as valuable today as when it was
originally published.

Almost every book on programming has something useful to say about
interface design. One practical book based on hard-won (难得的)
experience is
\bookname{Large-Scale C++ Software Design} by John Lakos (Addison-Wesley,
1996), which discusses how to build and manage truly large C++ programs.
David Hanson's \bookname{C Interfaces and Implementation}
(Addison-Wesley. 1997) is a good treatment for C programs.

Steve McConnell's \bookname{Rapid Development} (Microsoft Press, 1996) is
an excellent description of how to build software in teams, with an
emphasis on the role of prototyping.

There are several interesting books on the design of graphical user
interfaces, with a variety of different perspectives. We suggest
\bookname{Designing Visual Interfaces: Communication Oriented Techniques}
by Kevin Mullet and Darrell Sano (Prentice Hall, 1993), \bookname{Designing
the User Interface: Strategies for Effective Human-Computer Interaction} by
Ben Shneiderman (3rd edition, Addison-Wesley, 1997). \bookname{About Face:
The Essentials of User Interface Design} by Alan Cooper (IDG, 1995),
and \bookname{User Interface Design} by Harold Thimbleby (Addison-Wesley,
1990).
