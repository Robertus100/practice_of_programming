% vim: ts=4 sts=4 sw=4 et tw=75
% preface here.

\chapter{Preface}
\label{chap:preface}
The presentation is organized into nine chapters, ecah focusing on one
major aspect of programming practice.

Chapter \ref{chap:style} discusses programming style. Good style is so
important to good programming that we have chosen to cover it first.
Well-written programs are better than badly-written ones, they have fewer
errors and are easier to debug and to modify, so it is import to think
about style from the beginning. This chapter also introduces an important
theme in good programming, the use of idioms (惯用语法) appropriate to the
language being used.

Algorithms and data structures, the topics of Chapter \ref{chap:alds}, are
the core of the computer science curriculum (课程) and a major part of
programming courses. Since most readers will already be familiar with this
material, our treatment is intended as a brief review of the handful of
algorithms and data structures that show up in almost every program. More
complex algorithms and data structures usually evolve from these building
blocks, so one should master the basics.

Chapter \ref{chap:desipl} describes the design and implementation of a
small program that illustrates algorithm and data structure issues in a
realistic setting. The program is implemented in five language; comparing
the versions shows how the same data structures are handled in each, and
how expressiveness (表现力) and performance vary across a spectrum (系列)
languages.

Interfaces between users, programs, and parts of programs are fundamental
in programming and much of the success of software is determined by how
well interfaces are designed and implemented. Chapter \ref{chap:interface}
show the evolution of a small library for parsing a widely used data
format. Even though the example is small, it illustrates many of the
concerns of interfaces degisn: abstraction, information hiding, resource
management and error handling.

Much as we try to write a programs correctly the first time, bugs, and
therefore debugging are inevitable. Chapter \ref{chap:debug} gives
strategies and tactics (策略) for systematic and effective debugging. Among
the topics are the signatures of common bugs and the importance of
"numerology" (命理学), where patterns in debugging often indicates where a
problem lies.

Testing is an attempt to develop a reasonable assurance that a program is
wording correctly and that it stays correct as it evolves. The emphasis in
Chapter \ref{chap:testing} is on systematic testing by hand and machine.
Boundary condition tests probe at potential weak spots.
Mechanization (机械化) and test scaffolds (脚手架) make it easy to do
extensive testing with modest effort.  Stress tests provide a different
kind of testing than typical users do and ferret out (揪出) a different
class of bugs.

Computers are so fast and compilers are so good that many programs are fast
enough the day they are written. But others are too slow, or they use too
much memory, or both. Chapter \ref{chap:performance} presents an orderly
way to approach the task of making a program use resources efficiently, so
that the program remains correct and sound as it is made more efficient.

Chapter \ref{chap:portability} covers portability. Successful programs live
long enough that their environment changes, or they must be moved to new
system or new hardware or new countries. The goal of portability is to
reduce the maintenance of a program by minimizing the amount of change
neccessary to adapt it to a new environment.

Computing is righ in languages, not just the general-purpose ones that we
use for the bulk of programming, but also many specialized languages that
focus on narrow domains. Chapter \ref{chap:notation} presents serveral
examples of the importance of notation in computing, and shows how we can
use it to simplify programs, to guide implementations, and even to help us
write programs that write programs.

