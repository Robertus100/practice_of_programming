% vim: ts=4 sts=4 sw=4 et tw=75
\chapter{Testing}
\label{chap:testing}
\begin{quote}
    In ordinary computational practice by hand or by desk
% FIXME: The hyperlink of footnote doesn't response when click.
    mtichines\footnote{Unrecognized word}, it is the custom to check every
    step of the computation and, when an error is found, to localize it by
    a backward process starting from the first point where the error is
    noted.
\end{quote}
\begin{quotesrc}
% FIXME: The hyperlink of footnote doesn't response when click.
    Norbert Wiener, \bookname{Cybernetics}\footnote{控制论.}
\end{quotesrc}

Testing and debugging are often spoken as a single phrase but they are not
the same thing. To over-simplify, debugging is what you do when you know
that a program is broken. Testing is a determined, systematic attempt to
break a program that you think is working.

Edsger Dijkstra made the famous observation that testing can demonstrate
the presence of bugs, but not their absence. His hope is that programs can
be made correct by construction, so that there are no errors and thus no
need for testing. Though this is a fine goal, it is not yet realistic for
substantial (大量的) programs. So in this chapter we'll focus on how to
test to find errors rapidly, efficiently, and effectively.

Thinking about potential problems as you code is a good start. Systematic
testing, from easy tests to elaborate (详尽的) ones, helps ensure that
programs begin life working correctly and remain correct as they grow.
Automation helps to eliminate manual processes and encourages (促进)
extensive (多方面的) testing. And there are plenty of tricks of the trade
that programmers have learned from experience.

One way to write bug-free code is to generate it by a program. If some
programming task is understood so well that writing the code seems
mechanical, then it should be mechanized (机械化的). A common case occurs
when a program can be generated from a specification in some specialized
language. For example, we compile high-level languages into assembly code;
we use regular expressions to specify patterns of text; we use notations
like \verb'SUM(A1:A50)' to represent operations over a range of cells in a
spreadsheet. In such cases, if the generator or translator is correct and
if the specification is correct, the resulting program will be correct too.
We will cover this rich topic in more detail in Chapter
\ref{chap:notation}; in this chapter we will talk briefly about ways to
create tests from compact specifications.

\section{Test as You Write the Code}
\label{sec:test_as_you_write_the_code}

The earlier a problem is found, the better. If you think systematically
about what you are writing as you write it, you can verify simple
properties of the program as it is being constructed, with the result that
your code will have gone through one round of testing before it is even
compiled. Certain kinds of bugs never come to life.

\emph{Test code at its boundaries.} One technique is \textit{boundary
    condition} testing: as each small piece of code is written -- a loop or
a conditional statement, for example -- check right then that the condition
branches the right way or that the loop goes through the proper number of
times. This process is called boundary condition testing because you are
probing at the natural boundaries within the program and data, such as
non-existent or empty input, a single input item, an exactly full array,
and so on. The idea is that most bugs occur at boundaries. If a piece of
code is going to fail, it will likely fail at a boundary. Conversely
(相反), if it works at its boundaries, it's likely to work elsewhere too.

This fragment, modeled on \verb'fgets', reads characters until it finds a
newline or fills a buffer:
\begin{badcode}
    int     i;
    char    s[MAX];

    for (i = 0; (s[i] = getchar()) != '\n' && i < MAX - 1; ++i)
        ;
    s[--i] = '\0';
\end{badcode}
Imagine that you have just written this loop. Now simulate it mentally as
it reads a line. The first boundary to test is the simplest: an empty line.
If you start with a line that contains only a single newline, it's easy to
see that the loop stops on the first iteration with \verb'i' set to zero,
so the last line decrements \verb'i' to \verb'-1' and thus writes a null
byte into \verb's[-1]', which is before the beginning of the array.
Boundary condition testing finds the error.

If we rewrite the loop to use the conventional idiom for filling an array
with input characters, it looks likes this:
\begin{wellcode}
    for (i = 0; i < MAX - 1; i++)
        if ((s[i] = getchar()) == '\n')
            break;
    s[i] = '\0';
\end{wellcode}
Repeating the original boundary test, it's easy to verify that a line with
just a newline is handled correctly: \verb'i' is zero, the first input
character breaks out of the loop, and \verb"'\0'" is stored in \verb's[0]'.
Similar checking for inputs of one and two characters followed by a newline
give us confidence that the loop works near that boundary.

Boundary condition testing can catch lots of bugs, but not all of them. We
will return to this example in Chapter \ref{chap:portability}, where we
will show that it still has a portability bug.

The next step is to check input at the other boundary, where the array is
nearly full, exactly full, and over-full, particularly if the newline
arrives at the same time.  We won't write out the details here, but it's a
good exercise. Thinking about the boundaries raises the question of what to
do when the buffer fills before a \verb"'\n'" occurs; this gap (缺口) in
the specification should be resolved early, and testing boundaries helps to
identify it.

Boundary condition checking is effective for finding off-by-one (单一断接)
errors.  With practice, it becomes second nature, and many trivial (琐碎的)
bugs are eliminated before they ever happen.

\emph{Test pre- and post-conditions.} Another way to head off (拦截)
problems is to verify that expected or necessary properties hold before
(pre-condition) and after (post-condition) some piece of code executes.
Making sure that input values are within range is a common example of
testing a pre-condition. This function for computing the average of
\verb'n' elements in an array has a problem if \verb'n' is less than or
equal to zero:
\begin{badcode}
    double avg(double a[], int n)
    {
        int     i;
        double  sum;

        sum = 0.0;
        for (i = 0; i < n; i++)
            sum += a[i];
        return sum / n;
    }
\end{badcode}

What should \verb'avg' do if \verb'n' is zero? An array with no elements is
a meaningful concept although its average value is not. Should \verb'avg'
let the system catch the division by zero? Abort? Complain? Quietly return
some innocuous (无害的) value? What if \verb'n' is negative, which is
nonsensical (无意义) but not impossible?  As suggested in Chapter
\ref{chap:interface}, our preference would probably be to return 0 as the
average if \verb'n' is less than or equal to zero:
\begin{wellcode}
    return n <= 0 ? 0.0 : sum/n;
\end{wellcode}
but there's no single right answer.

The one guaranteed wrong answer is to ignore the problem. An article in the
November, 1998 \bookname{Scientific American} describes an incident (事件)
aboard (在...之上) the USS Yorktown, a guided-missile cruiser (导弹巡洋舰).
A crew member mistakenly entered a zero for a data value, which resulted in
a division by zero, an error that cascaded and eventually shut down the
ship's propulsion (推进) system. The Yorktown was dead in the water for a
couple of hours because a program didn't check for valid input.

\emph{Use assertions.} C and C++ provide an assertion facility in
\verb'assert.h' that encourages adding pre- and post-condition tests. Since
a failed assertion aborts the program, these are usually reserved for
situations where a failure is really unexpected and there's no way to
recover. We might augment (增加) the code above with an assertion before
the loop:
\begin{wellcode}
    assert(n > 0);
\end{wellcode}
If the assertion is violated, it will cause the program to abort with a
standard message:
\begin{wellcode}
    Assertion failed: n > 0, file avgtest.c, line 7
    Abort(crash)
\end{wellcode}
Assertions are particularly helpful for validating properties of interfaces
because they draw attention to inconsistencies between caller and callee
and may even indicate who's at fault. If the assertion that \verb'n' is
greater than zero fails when the function is called, it points the finger
(手指) at the caller rather than at \verb'avg' itself as the source of
trouble.  If an interface changes but we forget to fix some routine that
depends on it, an assertion may catch the mistake before it causes real
trouble.

\emph{Program defensively.} A useful technique is to add code to handle
"can't happen" cases, situations where it is not logically possible for
something to happen but (because of some failure elsewhere) it might
anyway. Adding a test for zero or negative array lengths to \verb'avg' was
one example. As another example, a program processing grades might expect
that there would be no negative or huge values but should check anyway:
\begin{wellcode}
    if (grade < 0 || grade > 100) /* can't happen */
        letter = '?';
    else if (grade >= 90)
        letter = 'A';
    else
        ...
\end{wellcode}
This is an example of defensive programming: making sure that a program
protects itself against incorrect use or illegal data. Null pointers, out
of range subscripts, division by zero, and other errors can be detected
early and warned about or deflected (转移).  Defensive programming (no pun
intended (没有双关语意)) might well have caught the zero-divide problem on
the Yorktown.

\emph{Check error returns.} One often-overlooked (常常被忽视的) defense is
to check the error returns from library functions and system calls. Return
values from input routines such as \verb'fread' and \verb'fscanf' should
always be checked for errors, as should any file open call such as
\verb'fopen'. If a read or open fails, computation cannot proceed
correctly.

Checking the return code from output functions like \verb'fprintf' or
\verb'fwrite' will catch the error that results from trying to write a file
when there is no space left on the disk.  It may be sufficient to check the
return value from \verb'fclose', which returns \verb'EOF' if any error
occurred during any operation, and zero otherwise.
\begin{wellcode}
    fp = fopen(outfile, "w");
    while (...)                 /* write output to outfile */
        fprintf(fp, ...);
    if (fclose(fp) == EOF) {    /* any errors? */
        /* some output error occurred */
    }
\end{wellcode}
Output errors can be serious. If the file being written is the new version
of a precious file, this check will save you from removing the old file if
the new one was not written successfully.

The effort of testing as you go is minimal and pays off handsomely
(优厚地).  Thinking about testing as you write a program will lead to
better code, because that's when you know best what the code should do. If
instead you wait until something breaks, you will probably have forgotten
how the code works. Working under pressure, you will need to figure it out
again, which takes time, and the fixes will be less thorough and more
fragile because your refreshed understanding is likely to be incomplete.

\begin{exercise}
    Check out these examples at their boundaries, then fix them as
    necessary according to the principles of style in Chapter
    \ref{chap:style} and the advice in this chapter.
    \begin{enumerate}
        \item This is supposed to compute factorials:
        \begin{badcode}
            int factorial(int n)
            {
                int fac;
                fac = 1;
                while (n--)
                    fac *= n;
                return fac;
            }
        \end{badcode}
        \item This is supposed to print the characters of a string one per
            line:
        \begin{badcode}
            i = 0;
            do {
                putchar(s[i++];
                putchar('\n');
            } while (s[i] != '\0');
        \end{badcode}
        \item This is meant to copy a string from source to destination:
        \begin{badcode}
            void strcpy(char *dest, char *src)
            {
                int i;

                for (i = 0; src[i] != '\0'; i++)
                    dest[i] = src[i];
            }
        \end{badcode}
        \item Another string copy, which attempts to copy \verb'n'
            characters from \verb's' to \verb't':
        \begin{badcode}
            void strncpy(char *s, char *t, int n)
            {
                while (n > 0 && *s != '\0') {
                    *t = *s;
                    t++;
                    s++;
                    n--;
                }
            }
        \end{badcode}
        \item A numerical comparison:
        \begin{badcode}
            if (i > j)
                printf("%d is greater than %d.\n", i, j);
            else
                printf("%d is smaller than %d.\n", i, j);
        \end{badcode}
        \item A character class test:
        \begin{badcode}
            if (c >= 'A' && c <= 'Z') {
                if (c >= 'L')
                    cout << "first half of alphabet";
                else
                    cout << "second half of alphabet";
            }
        \end{badcode}
    \end{enumerate}
\end{exercise}

\begin{exercise}
    As we are writing this book in late 1998, the Year 2000 problem looms
    (若隐若现) as perhaps the biggest boundary condition problem ever.
    \begin{enumerate}
        \item What dates would you use to check whether a system is likely
            to work in the year 2000? Supposing that test are expensive to
            perform, in what order would you do your tests after trying
            January 1, 2000 itself?
        \item How would you test the standard function \verb'ctime', which
            returns a string representation of the date in this form:
        \begin{wellcode}
            Fri Dec 31 23:58:27 EST 1999\n\0
        \end{wellcode}
        Suppose your program calls \verb'ctime'. How would you write your
        code to defend against a flawed (缺陷的) implementation?
        \item Describe how you would test a calendar program that prints
            output like this:
        \begin{wellcode}
             July 2015        
        Su Mo Tu We Th Fr Sa  
                 1  2  3  4  
        5  6  7  8  9  10 11  
        12 13 14 15 16 17 18  
        19 20 21 22 23 24 25  
        26 27 28 29 30 31     
        \end{wellcode}
        \item What other time boundaries can you think of in systems that
            you use, and how would you test to see whether they are handled
            correctly?
    \end{enumerate}
\end{exercise}

\section{Systematic Testing}
\label{sec:systematic_testing}

It's important to test a program systematically so you know at each step
what you are testing and what results you expect. You need to be orderly so
you don't overlook (忽略) anything, and you must keep records so you know
how much you have done.

\emph{Test incrementally.} Testing should go hand in hand (联合) with
program construction. A "big bang" (大撞击) where one writes the whole
program, then tests it all at once, is much harder and more time-consuming
than an incremental approach. Write part of a program, test it, add some
more code, test that, and so on. If you have two packages that have been
written and tested independently, test that they work together when you
finally connect them.

For instance, when we were testing the CSV programs in Chapter
\ref{chap:interface}, the first step was to write just enough code to read
the input; this let us validate input processing.  The next step was to
split input lines at commas. Once these parts were working, we moved on to
fields with quotes, and then gradually worked up to testing everything.

\emph{Test simple parts first.} The incremental approach also applies to
how you test features. Tests should focus first on the simplest and most
commonly executed features of a program; only when those are working
properly should you move on. This way, at each stage, you expose more to
testing and build confidence that basic mechanisms are working correctly.
Easy tests find the easy bugs. Each test does the minimum to ferret out
(查获) the next potential problem. Although each bug is harder to trigger
than its predecessor, it is not necessarily harder to fix.

In this section, we'll talk about ways to choose effective tests and in
what order to apply them; in the next two sections, we'll talk about how to
mechanize (使机械化) the process so that it can be carried out efficiently.
The first step, at least for small programs or individual functions, is an
extension of the boundary condition testing that we described in the
previous section: systematic testing of small cases.

Suppose we have a function that performs binary search in an array of
integers.  We would begin with these tests, arranged in order of increasing
complexity:
\begin{itemize}
    \item search an array with noelements
    \item search an array with one element and a trial (试验的) value this
        is
        \begin{itemize}
            \item less than the single element in the array
            \item equal to the single element
            \item greater than the single element
        \end{itemize}
    \item search an array with two elements and trial values that
        \begin{itemize}
            \item check all five possible positions
        \end{itemize}
    \item check behavior with duplicate elements in the array and trial
        values
        \begin{itemize}
            \item less than the value in the array
            \item equal to the value
            \item greater than the value
        \end{itemize}
    \item search an array with three elements as with two elements
    \item search an array with four elements as with two and three
\end{itemize}
If the function gets past this unscathed (未受损伤的), it's likely to be in
good shape, but it could still be tested further.

This set of tests is small enough to perform by hand, but it is better to
create a test scaffold (脚手架) to mechanize the process. The following
driver program is about as simple as we can manage. It reads input lines
that contain a key to search for and an array size; it creates an array of
that size containing values \verb'1, 3, 5, ...', and it searches the array
for the key.
\begin{wellcode}
    /* bintest main: scaffold for testing binsearch */
    int main(void)
    {
        int i, key, nelem, arr[1000];

        while (scanf("%d %d", &key, &nelem) != EOF) {
            for (i = 0; i < nelem; i++)
                arr[i] = 2 * i + 1;
            printf("%d\n", binsearch(key, arr, nelem));
        }
        return 0;
    }
\end{wellcode}
This is simpleminded but it shows that a useful test scaffold need not be
big, and it is easily extended to perform more of these tests and require
less manual intervention (介入).

\emph{Know what output to expect.} For all tests, it's necessary to know
what the right answer is; if you don't, you're wasting your time. This
might seem obvious, since for many programs it's easy to tell whether the
program is working. For example, either a copy of a tile (瓦片) is a copy
or it isn't. The output from a sort is sorted or it isn't; it must also be
a permutation (排列) of the original input.

Most programs are more difficult to characterize (描绘特性) -- compilers
(does the output properly translate the input?), numerical algorithms (is
the answer within error tolerance?), graphics (are the pixels in the right
places?), and so on. For these, it's especially important to validate the
output by comparing it with known values.
\begin{itemize}
    \item To test a compiler, compile and run the test files. The test
        programs should in turn generate output, and their results should
        be compared to known ones.
    \item To test a numerical program, generate test cases that explore the
        edges of the algorithm, trivial (琐细的) cases as well as hard
        ones. Where possible, write code that verifies that output
        properties are sane (合理的). For example, the output of a
        numerical integrator can be tested for continuity (连续性), and for
        agreement with closed-form (封闭形式) solutions.
    \item To test a graphics program, it's not enough to see if it can draw
        a box; instead read the box back from the screen and check that its
        edges are exactly where they should be.
\end{itemize}

If the program has an inverse (相反的), check that its application recovers
the input.  Encryption and decryption are inverses, so if you encrypt
something and can't decrypt it, something is wrong. Similarly, lossless
compression and expansion algorithms should be inverses. Programs that
bundle (打包) files together should extract them unchanged. Sometimes there
are multiple methods for inversion: check all combinations.

\emph{Verify conservation (保存) properties.} Many programs preserve some
property of their inputs. Tools like \texttt{wc} (count lines, words, and
characters) and \texttt{sum} (compute a checksum) can verify that outputs
are of the same size, have the same number of words, contain the same bytes
in some order, and the like. Other programs compare files for identity
(\texttt{cmp}) or report differences (\texttt{diff}). These programs or
similar ones are readily (轻易的) available for most environments, and are
well worth acquiring.

A byte-frequency program can be used to check for conservation of data and
also to spot anomalies (反常) like non-text characters in supposedly
text-only files; here's a version that we call \texttt{freq}:
\begin{wellcode}
    #include <stdio.h>
    #include <ctype.h>
    #include <limits.h>

    unsigned long count[UCHAR_MAX+1];

    /* freq main: display byte frequency counts */
    int main(void)
    {
        int c;

        while ((c = getchar()) != EOF)
            count[c]++;
        for (c = 0; c <= UCHAR_MAX; c++)
            if (count[c] != 0)
                printf("%.2x %c %lu\n",
                        c, isprint(c) ? c : '-', count[c]);
        return 0;
    }
\end{wellcode}

Conservation properties can be verified within a program, too. A function
that counts the elements in a data structure provides a trivial (平常的)
consistency check. A hash table should have the property that every element
inserted into it can be retrieved.  This condition is easy to check with a
function that dumps the contents of the table into a file or an array. At
any time, the number of insertions into a data structure minus the number
of deletions must equal the number of elements contained, a condition that
is easy to verify.

\emph{Compare independent implementations.} Independent implementations of
a library or program should produce the same answers. For example, two
compilers should produce programs that behave the same way on the same
machine, at least in most situations.

Sometimes an answer can be computed in two different ways, or you might be
able to write a trivial version of a program to use as a slow but
independent comparison. If two unrelated programs get the same answers,
there is a good chance that they are correct; if they get different
answers, at least one is wrong.

One of the authors once worked with another person on a compiler for a new
machine. The work of debugging the code generated by the compiler was
split: one person wrote the software that encoded instructions for the
target machine, and the other wrote the disassembler for the debugger. This
meant that any error of interpretation or implementation of the instruction
set was unlikely to be duplicated between the two components. When the
compiler miscoded an instruction, the disassembler was sure to notice. All
the early output of the compiler was run through the disassembler and
verified against the compiler's own debugging printouts. This strategy
worked very well in practice, instantly catching mistakes in both pieces.
The only difficult, protracted (拖延的) debugging occurred when both people
interpreted an ambiguous phrase in the architecture description in the same
incorrect way.

\emph{Measure test coverage (覆盖范围).} One goal of testing is to make
sure that every statement of a program has been executed sometime during
the sequence of tests; testing cannot be considered complete unless every
line of the program has been exercised by at least one test. Complete
coverage is often quite difficult to achieve. Even leaving aside "can't
happen" statements, it is hard to use normal inputs to force a program to
go through particular statements.

There are commercial tools for measuring coverage. Profilers (探查程序),
often included as part of compiler suites, provide a way to compute a
statement frequency count for each program statement that indicates the
coverage achieved by specific tests.

We tested the \texttt{Markov} program of Chapter \ref{chap:desipl} with a
combination of these techniques. The last section of this chapter describes
those tests in detail.

\begin{exercise}
    Describe how you would test \texttt{freq}.
\end{exercise}

\begin{exercise}
    Design and implement a version of \texttt{freq} that measures the
    frequencies of other types of data values, such as 32-bit integers or
    floating-point numbers. Can you make one version of the program handle
    a variety of types elegantly?
\end{exercise}

\section{Test Automation}
\label{sec:test_automation}

It's tedious and unreliable to do much testing by hand; proper testing
involves lots of tests, lots of inputs, and lots of comparisons of outputs.
Testing should therefore be done by programs, which don't get tired or
careless. It's worth taking the time to write a script or trivial program
that encapsulates all the tests, so a complete test suite can be run by
(literally (从字面上) or figuratively (比喻)) pushing a single button. The
easier a test suite is to run, the more often you'll run it and the less
likely you'll skip it when time is short. We wrote a test suite that
verifies all the programs we wrote for this book, and ran it every time we
made changes; parts of the suite ran automatically after each successful
compilation.

\emph{Automate regression testing.} The most basic form of automation is
regression testing, which performs a sequence of tests that compare the new
version of something with the previous version. When fixing problems,
there's a natural tendency to check only that the fix works; it's easy to
overlook the possibility that the fix broke something else. The intent of
regression testing is to make sure that the behavior hasn't changed except
in expected ways.

Some systems are rich in tools that help with such automation; scripting
languages allow us to write short scripts to run test sequences. On Unix,
file comparators like \texttt{diff} and \texttt{cmp} compare outputs;
\texttt{sort} brings common elements together; \texttt{grep} filters test
outputs; \texttt{wc}, \texttt{sum}, and \texttt{freq} summarize outputs.
Together, these make it easy to create ad hoc (特别定制地) test scaffolds,
maybe not enough for large programs but entirely adequate for a program
maintained by an individual or a small group.

Here is a script for regression testing a killer application program called
\texttt{ka}. It runs the old version (\verb'old_ka') and the new version
(\verb'new_ka') for a large number of different test data files, and
complains about each one for which the outputs are not identical. It is
written for a Unix shell but could easily be transcribed to Perl or other
scripting language:
\begin{wellcode}
    for i in ka_data.*              # loop over test data files
    do
        old_ka $i > out1            # run the old version
        new_ka $i > out2            # run the new version
        if ! cmp -s out1 out2       # compare output files
        then
            echo $BAD               # different: print error message
        fi
    done
\end{wellcode}

A test script should usually run silently, producing output only if
something unexpected occurs, as this one does. We could instead choose to
print each file name as it is being tested, and to follow it with an error
message if something goes wrong. Such indications of progress help to
identify problems like an infinite loop or a test script that is failing to
run the right tests, but the extra chatter (喋喋不休) is annoying if the
tests are running properly.

The \texttt{-s} argument causes \texttt{cmp} to report status but produce
no output. If the files compare equal, \texttt{cmp} returns a true status,
\texttt{!cmp} is false, and nothing is printed. If the old and new outputs
differ, however, \texttt{cmp} returns false and the file name and a warning
are printed.

There is an implicit assumption in regression testing that the previous
version of the program computes the right answer. This must be carefully
checked at the beginning of time, and the invariant (不变的) scrupulously
(审慎地) maintained. If an erroneous answer ever sneaks (偷偷地) into a
regression test, it's very hard to detect and everything that depends on it
will be wrong thereafter (其后). It's good practice to check the regression
test itself periodically to make sure it is still valid.

\emph{Create self-contained tests.} Self-contained tests that carry their
own inputs and expected outputs provide a complement (补充) to regression
tests. Our experience testing Awk may be instructive (有益的). Many
language constructions are tested by running specified inputs through tiny
programs and checking that the right output is produced. The following part
of a large collection of miscellaneous tests verifies one tricky (狡猾的)
increment expression. This test runs the new version of Awk
(\texttt{newawk}) on a short Awk program to produce output in one file,
writes the correct output to another file with \texttt{echo}, compares the
files, and reports an error if they differ.
\begin{wellcode}
    # field increment test: $i++ means ($i)++, not $(i++)

    echo 3 5 | newawk '{i = 1; print $i++; print $1, i}' > out1
    echo '3
    4 1' > out2     # correct answer
    if ! cmp -s out1 out2   # outputs are different
    then
        echo 'BAD: field increment test failed'
    fi
\end{wellcode}
The first comment is part of the test input; it documents what the test if
testing.

Sometimes it is possible to construct a large number of tests with modest
effort. For simple expressions, we created a small, specialized language
for describing tests, input data, and expected outputs. Here is a short
sequence that tests some of the ways that the numeric value \texttt{1} can
be represented in Awk:
\begin{wellcode}
    try {if ($1 == 1) print "yes"; else print "no"}
    1       yes
    1.0     yes
    1E0     yes
    0.1E1   yes
    10E-1   yes
    01      yes
    +1      yes
    10E-2   no
    10      no
\end{wellcode}
The first line is a program to be tested (everything after the word
\texttt{try}).  Each subsequent line is a set of inputs and the expected
output, separated by tabs.  The first test says that if the first input
field is \texttt{1} the output should be yes. The first seven tests should
all print yes and the last two tests should print no.

An Awk program (what else?) converts each test into a complete Awk program,
then runs each input through it, and compares actual output to expected
output; it reports only those cases where the answer is wrong.

Similar mechanisms are used to test the regular expression matching and
substitution commands. A little language for writing tests makes it easy to
create a lot of them; using a program to write a program to test a program
has high leverage (杠杆效率). (Chapter \ref{chap:notation} has more to say
about little languages and the use of programs that write programs.)

Overall, there are about a thousand tests for Awk; the whole set can be run
with a single command, and if everything goes well, no output is produced.
Whenever a feature is added or a bug is fixed, new tests are added to
verify correct operation. When ever the program is changed, even in a
trivial (琐碎的) way, the whole test suite is run; it takes only a few
minutes. It sometimes catches completely unexpected errors, and has saved
the authors of Awk from public embarrassment many times.

What should you do when you discover an error? If it was not found by an
existing test, create a new test that does uncover the problem and verify
the test by running it with the broken version of the code. The error may
suggest further tests or a whole new class of things to check. Or perhaps
it is possible to add defenses to the program that would catch the error
internally.

Never throw away a test. It can help you decide whether a bug report is
valid or describes something already fixed. Keep a record of bugs, changes,
and fixes; it will help you identify old problems and fix new ones. In most
commercial programming shops (工作室), such records are mandatory. For your
personal programming, they are a small investment that will pay off
repeatedly.

\begin{exercise}
    Design a test suite for \texttt{printf}, using as many mechanical aids
    as possible.
\end{exercise}

\section{Test Scaffolds}
\label{sec:test_scaffolds}

Our discussion so far is based largely on testing a single stand-alone
program in its completed form. This is not the only kind of test
automation, however, nor is it the most likely way to test parts of a big
program during construction, especially if you are part of a team. Nor is
it the most effective way to test small components that are buried in
something larger.

To test a component in isolation, it's usually necessary to create some
kind of framework or scaffold that provides enough support and interface to
the rest of the system that the part under test will run. We showed a tiny
example for testing binary search earlier in this chapter.

It's easy to build scaffolds for testing mathematical functions, string
functions, sort routines, and so on, since the scaffolding is likely to
consist mostly of setting up input parameters, calling the functions to be
tested, then checking the results. It's a bigger job to create scaffolding
for testing a partly-completed program.

To illustrate, we'll walk through building a test for \texttt{memset}, one
of the \verb'mem...' functions in the C/C++ standard library. These
functions are often written in assembly language for a specific machine,
since their performance is important.  The more carefully tuned they are,
however, the more likely they are to be wrong and thus the more thoroughly
they should be tested.

The first step is to provide the simplest possible C versions that are
known to work; these provide a benchmark for performance and, more
important, for correctness. To move to a new environment, one carries the
simple versions and uses them until the tuned ones are working.

The function \verb'memset(s, c, n)' sets \verb'n' bytes of memory to the
byte \verb'c', starting at address \verb's', and returns \verb's'. This
function is easy if speed is not an issue:
\begin{wellcode}
    /* memset: set first n bytes of s to c */
    void *memset(void *s, int c, size_t n)
    {
        size_t  i;
        char    *p;

        p = (char *)s;
        for (i = 0; i < n; i++)
            p[i] = c;
        return s;
    }
\end{wellcode}
But when speed is an issue, tricks like writing full words of 32 or 64 bits
at a time are used. These can lead to bugs, so extensive (广泛的) testing
is mandatory.

Testing is based on a combination of exhaustive (彻底的) and
boundary-condition checks at likely points of failure. For \texttt{memset},
the boundaries include obvious values of \texttt{n} such as zero, one and
two, but also values that are powers of two or nearby values. including
both small ones and large ones like $2^{16}$, which corresponds to a natural
boundary in many machines, a 16-bit word. Powers of two deserve attention
because one way to make \texttt{memset} faster is to set multiple bytes at
one time; this might be done by special instructions or by trying to store
a word at a time instead of a byte.  Similarly, we want to check array
origins with a variety of alignments in case there is some error based on
starting address or length. We will place the target array inside a larger
array, thus creating a buffer zone or safety margin on each side and giving
us an easy way to vary the alignment.

We also want to check a variety of values for \texttt{c}, including zero,
\texttt{0x7F} (the largest signed value, assuming 8-bit bytes),
\texttt{0x80} and \texttt{0xFF} (probing at potential errors involving
signed and unsigned characters), and some values much bigger than one byte
(to be sure that only one byte is used). We should also initialize memory
to some known pattern that is different from any of these character values
so we can check whether \texttt{memset} wrote outside the valid area.

We can use the simple implementation as a standard of comparison in a test
that allocates two arrays, then compares behaviors on combinations of
\texttt{n}, \texttt{c} and offset within the array:
\\ \\
\indent \texttt{big} = maximum left margin + maximum \texttt{n} + maximum
right margin \\
\indent \texttt{s0 = malloc(big)} \\
\indent \texttt{s1 = malloc(big)} \\
\indent for each combination of test parameters \texttt{n}, \texttt{c}, and
\texttt{offset}: \\
\indent \indent set all of \texttt{s0} and \texttt{s1} to knwon pattern \\
\indent \indent run slow \texttt{memset(s0 + offset, c, n)} \\
\indent \indent run fast \texttt{memset(s1 + offset, c, n)} \\
\indent \indent check return values \\
\indent \indent copare all of \texttt{s0} and \texttt{s1} byte by byte \\
An error that causes \texttt{memset} to write outside the limits of its
array is most likely to affect bytes near the beginning or the end of the
array, so leaving a buffer zone makes it easier to see damaged bytes and
makes it less likely that an error will overwrite some other part of the
program. To check for writing out of bounds, we compare all the bytes of
\texttt{s0} and \texttt{s1}, not just the \texttt{n} bytes that should be
written.

Thus a reasonalbe set of tests might include all combinations of:
\begin{wellcode}
    offset = 10, 11, ..., 20
    c = 0, 1, 0x7F, 0x80, 0xFF, 0x11223344
    n = 0, 1, 2, 3, 4, 5, 7, 8, 9, 15, 16, 17,
        31, 32, 33, ..., 65535, 65536, 65537
\end{wellcode}
The values of \texttt{n} would include at least $2^i - 1$, $2^i$ and
$2^i + 1$ for $i$ from 0 to 16.

These values should not be wired into the main part of the test scaffold,
but should appear in arrays that might be created by hand or by program.
Generating them automatically is better; that makes it easy to specify more
powers of two or to include more offsets and more characters.

These tests will give \texttt{memset} a thorough workout (试验) yet cost
very little time even to create, let alone run, since there are fewer than
3500 cases for the values above. The tests are completely portable, so they
can be carried to a new environment as necessary.

As a warning, consider this story. We once gave a copy of a \texttt{memset}
tester to someone developing an operating system and libraries for a new
processor.  Months later, we (the authors of the original test) started
using the machine and had a large application fail its test suite. We
traced the problem to a subtle (微妙的) bug involving sign extension in the
assembly language implementation of \texttt{memset}. For reasons unknown,
the library implementer had changed the \texttt{memset} tester so it did
not check values of \texttt{c} above \texttt{0x7F}. Of course, the bug was
isolated by running the original, working tester, once we realized that
\texttt{memset} was a suspect.

Functions like \texttt{memset} are susceptible (容许...的) to exhaustive
tests because they are simple enough that one can prove that the test cases
exercise all possible execution paths through the code, thus giving
complete coverage. For example, it is possible to test \texttt{memmove} for
all combinations of overlap, direction, and alignment. This is not
exhaustive in the sense of testing all possible copy operations, but it is
an exhaustive test of representatives (代表) of each kind of distinct input
situation.

As in any testing method, test scaffolds need the correct answer to verify
the operations they are testing. An important technique, which we used in
testing \texttt{memset}, is to compare a simple version that is believed
correct against a new version that may be incorrect. This can be done in
stages, as the following example shows.

One of the authors implemented a raster (光栅) graphics library involving
an operator that copied blocks of pixels from one image to another.
Depending on the parameters, the operation could be a simple memory copy,
or it could require converting pixel values from one color space to
another, or it could require "tiling" (盖瓦) where the input was copied
repeatedly throughout a rectangular area, or combinations of these and
other features. The specification of the operator was simple, but an
efficient implementation would require lots of special code for the many
cases. To make sure all that code was right demanded a sound (可靠的)
testing strategy.

First, simple code was written by hand to perform the correct operation for
a single pixel. This was used to test the library version's handling of a
single pixel. Once this stage was working, the library could be trusted for
single-pixel operations.

Next, hand-written code used the library a pixel at a time to build a very
slow version of the operator that worked on a single horizontal row of
pixels, and that was compared with the library's much more efficient
handling of a row. With that working, the library could be trusted for
horizontal lines.

This sequence continued, using lines to build rectangles, rectangles to
build tiles (瓦片), and so on. Along the way, many bugs were found,
including some in the tester itself, but that's part of the effectiveness
of the method: we were testing two independent implementations, building
confidence in both as we went. If a test failed, the tester printed out a
detailed analysis to aid understanding what went wrong, and also to verify
that the tester was working properly itself.

As the library was modified and ported over the years, the tester
repeatedly proved invaluable for finding bugs.

Because of its layer-by-layer approach, this tester needed to be run from
scratch (擦除) each time, to verify its own trust of the library.
Incidentally, the tester was not exhaustive, but probabilistic (概率性的):
it generated random test cases which, for long enough runs, would
eventually explore every cranny (缝隙) of the code. With the huge number of
possible test cases, this strategy was more effective than trying to
construct a thorough test set by hand, and much more efficient than
exhaustive testing.

\begin{exercise}
    Create the test scaffold for \verb'memset' along the lines that we
    indicated.
\end{exercise}

\begin{exercise}
    Create tests for the rest of the \verb'mem...' family.
\end{exercise}

\begin{exercise}
    Specify a testing regime (体制) for numerical routines like
    \verb'sqrt', \verb'sin', and so on, as found in \verb'math.h'. What
    input values make sense? What independent checks can be performed?
\end{exercise}

\begin{exercise}
    Define mechanisms for testing the functions of the C \verb'str...'
    family, like \verb'strcmp'. Some of these functions, especially
    tokenizers like \verb'strtok' and \verb'strcspn', are significantly
    more complicated than the \verb'mem...' family, so more sophiscated
    test will be called for.
\end{exercise}

\section{Stress Tests}
\label{sec:stress_tests}

High volumes of machine-generated input are another effective testing
technique.  Machine-generated input stresses programs differently than
input written by people does. Higher volume in itself tends to break things
because very large inputs cause overflow of input buffers, arrays, and
counters, and are effective at finding unchecked fixed-size storage within
a program. People tend to avoid "impossible" cases like empty inputs or
input that is out of order or out of range, and are unlikely to create very
long names or huge data values. Computers, by contrast, produce output
strictly according to their programs and have no idea of what to avoid.

To illustrate, here is a single line of output produced by the Microsoft
Visual C++ Version 5.0 compiler while compiling the C++ STL implementation
of \texttt{markov}; we have edited the line so it fits:
\begin{wellcode}
    xtree(114) : warning C4786: 'std::_Tree<std::deque<std::
    basic_string<char, std::char_traits<char>,std::allocator
    <char>>,std::allocator<std::basic_string<char,std::
    ... 1420 characters ommited
    allocator<char>>>>>::iterator' : identifer was
    truncated to '255' characters in the debug information
\end{wellcode}
The compiler is warning us that it has generated a variable name that is a
remarkable 1594 characters long but that only 255 characters have been
preserved as debugging information. Not all programs defend themselves
against such unusually long strings.

Random inputs (not necessarily legal) are another way to assault (攻击) a
program in the hope of breaking something. This is a logical extension of
"people don't do that" reasoning. For example, some commercial C compilers
are tested with randomly-generated but syntactically valid programs. The
trick is to use the specification of the problem -- in this case, the C
standard -- to drive a program that produces valid but bizarre (奇异的)
test data.

Such tests rely on detection by built-in checks and defenses in the
program, since it may not be possible to verify that the program is
producing the right output; the goal is more to provoke (诱导) a crash or a
"can't happen" than to uncover (揭开) straightforward errors. It's also a
good way to test that error-handling code works. With sensible input, most
errors don't happen and code to handle them doesn't get exercised: by
nature, bugs tend to hide in such corners. At some point, though, this kind
of testing reaches diminishing (逐渐减小的) returns: it finds problems that
are so unlikely to happen in real life they may not be worth fixing.

Some testing is based on explicitly malicious (恶意的) inputs. Security
attacks often use big or illegal inputs that overwrite precious data; it is
wise to look for such weak spots. A few standard library functions are
vulnerable to this sort of attack. For instance, the standard library
function \verb'gets' provides no way to limit the size of an input line, so
it should \emph{never} be used; always use
\verb'fgets(buf, sizeof(buf), stdin)'
instead. A bare (无遮蔽的) \verb'scanf("%s", buf)' doesn't limit
the length of an input line either; it should therefore usually be used
with an explicit length, such as \verb'scanf("%20s", buf)'. In Section
\ref{sec:building_the_data_structure_in_c} we showed how to address this
problem for a general buffer size.

Any routine that might receive values from outside the program, directly or
indirectly, should validate its input values before using them. The
following program from a textbook is supposed to read an integer typed by a
user, and warn if the integer is too long. Its goal is to demonstrate how
to overcome the \texttt{gets} problem, but the solution doesn't always
work.
\begin{badcode}
    #define MAXNUM 10

    int main(void)
    {
        char    num[MAXNUM];

        memset(num, 0, sizeof(num));
        printf("Type a number: ");
        gets(num);
        if (num[MAXNUM - 1] != 0)
            printf("Number too bit.\n");
        /* ... */
    }
\end{badcode}
If the input number is ten digits long, it will overwrite the last zero in
array num with a non-zero value, and in theory this will be detected after
the return from gets. Unfortunately, this is not sufficient. A malicious
attacker can provide an even longer input string that overwrites some
critical value, perhaps the return address for the call, so the program
never returns to the \verb'if' statement but instead executes something
nefarious (恶毒的). Thus this kind of unchecked input is a potential
security problem.

Lest (免得) you think that this is an irrelevant textbook example, in July,
1998 an error of this form was uncovered in several major electronic mail
programs. As the \textit{New York Times} reported,
\begin{quotation}
    The security hole is caused by what is known as a "buffer overflow
    error." Programmers are supposed to include code in their software to
    check that incoming data are of a safe type and that the units are
    arriving at the right length. If a unit of data is too long, it can
    overrun the "buffer" -- the chunk of memory set aside to hold it. In
    that case, the E-mail program will crash, and a hostile (怀有敌意的)
    programmer can trick the computer into running a malicious program in
    its place.
\end{quotation}
This was also one of the attacks in the famous "Internet Worm" incident of
1988.

Programs that parse HTML forms can also be vulnerable to attacks that store
very long input strings in small arrays:
\begin{badcode}
    static char query[1024];

    char *read_form(void)
    {
        int qsize;

        qsize = atoi(getenv("CONTENT_LENGTH"));
        fread(query, qsize, 1, stdin);
        return query;
    }
\end{badcode}
The code assumes that the input will never be more than 1024 bytes long so,
like \verb'gets', it is open to an attack that overflows its buffer.

More familiar kinds of overflow can cause trouble, too. If integers
overflow
silently, the result can be disastrous. Consider an allocation like
\begin{badcode}
    char *p;
    p = (char *)malloc(x * y * z);
\end{badcode}

If the product of \verb'x', \verb'y', and \verb'z' overflows, the call to
\verb'malloc' might produce a reasonable-sized array, but \verb'p[x]' might
refer to memory outside the allocated region.  Suppose that \verb'int's are
16 bits and \verb'x', \verb'y', and \verb'z' are each 41. Then \verb'x*y*z'
is 68921, which is 3385 module $2^{16}$. So the call to \verb'malloc'
allocates only 3385 bytes; any reference with a subscript beyond that value
will be out of bounds.

Conversion between types is another source of ovefflow, and catching the
error may not be good enough. The Ariane 5 rocket exploded on its maiden
(首次) flight in June, 1996 because the navigation package was inherited
from the Ariane 4 without proper testing. The new rocket flew faster,
resulting in larger values of some variables in the navigation software.
Shortly after launch, an attempt to convert a 64-bit floating-point number
into a 16-bit signed integer generated an overflow. The error was caught,
but the code that caught it elected to shut down the subsystem. The rocket
veered off course (冲出航线) and exploded. It was unfortunate that the code
that failed generated inertial (惯性的) reference information useful only
before lift-off (起飞); had it been turned off at the moment of launch,
there would have been no trouble.

On a more mundane (世俗的) level, binary inputs sometimes break programs
that expect text inputs, especially if they assume that the input is in the
7-bit ASCII character set.  It is instructive (有益的) and sometimes
sobering (合理的) to pass binary input (such as a compiled program) to an
unsuspecting program that expects text input.

Good test cases can often be used on a variety of programs. For example,
any program that reads files should be tested on an empty file. Any program
that reads text should be tested on binary files. Any program that reads
text lines should be tested on huge lines and empty lines and input with no
newlines at all. It's a good idea to keep a collection of such test files
handy, so you can test any program with them without having to recreate the
tests. Or write a program to create test files upon demand.

When Steve Bourne was writing his Unix shell (which came to be known as the
Bourne shell), he made a directory of 254 files with one-character names,
one for each byte value except \verb"'\0'" and slash, the two characters
that cannot appear in Unix file names. He used that directory for all
manner (方式) of tests of pattern-matching and tokenization. (The test
directory was of course created by a program.) For years afterwards, that
directory was the bane (祸根) of file-tree-walking programs; it tested them
to destruction (破坏).

\begin{exercise}
    Try to create a file that will crash your favorite text editor,
    compiler or other program.
\end{exercise}

\section{Tips for Testing}
\label{sec:tips_for_testing}

Experienced testers use many tricks and techniques to make their work more
productive; this section includes some of our favorites.

Programs should check array bounds (if the language doesn't do it for
them), but the checking code might not be tested if the array sizes are
large compared to typical input. To exercise the checks, temporarily make
the array sizes very small, which is easier than creating large test cases.
We used a related trick in the array-growing code in Chapter
\ref{chap:alds} and in the CSV library in Chapter \ref{chap:interface}. In
fact, we left the tiny initial values in place, since the additional
startup cost is negligible (可以忽略的).

Make the hash function return a constant, so every element gets installed
in the same hash bucket. This will exercise the chaining mechanism; it also
provides an indication (指示) of worst-case performance.

Write a version of your storage allocator that intentionally fails early,
to test your code for recovering from out-of-memory errors. This version
returns \verb'NULL' after 10 calls:
\begin{wellcode}
    /* testmalloc: returns NULL after 10 calls */
    void *testmalloc(size_t n)
    {
        static int count = 0;
        if (++count > 10)
            return NULL;
        else
            return malloc(n);
    }
\end{wellcode}

Before you ship (发布) your code, disable testing limitations that will
affect performance.  We once tracked down a performance problem in a
production compiler to a hash function that always returned zero because
testing code had been left installed.

Initialize arrays and variables with some distinctive (有特色的) value,
rather than the usual default of zero; then if you access out of bounds or
pick up an uninitialized variable, you are more likely to notice it. The
constant \verb'0xDEADBEEF' is easy to recognize in a debugger; allocators
sometimes use such values to help catch uninitialized data.

Vary your test cases, especially when making small tests by hand -- it's
easy to get into a rut (车辙) by always testing the same thing, and you may
not notice that something else has broken.

Don't keep on implementing new features or even testing existing ones if
there are known bugs; they could be affecting the test results.

Test output should include all input parameter settings, so the tests can
be reproduced exactly. If your program uses random numbers, have a way to
set and print the starting seed, independent of whether the tests
themselves are random. Make sure that test inputs and corresponding outputs
are properly identified, so they can be understood and reproduced.

It's also wise to provide ways to make the amount (总数) and type of output
controllable when a program is run; extra output can help during testing.

Test on multiple machines, compilers, and operating systems. Each
combination potentially reveals errors that won't be seen on others, such
as dependencies on byte-order, sizes of integers, treatment of null
pointers, handling of carriage return and newline, and specific properties
of libraries and header files. Testing on multiple machines also uncovers
problems in gathering the components of a program for shipment and, as we
will discuss in Chapter \ref{chap:portability}, may reveal unwitting
(没有意识到的) dependencies on the development environment.

We will discuss performance testing in Chapter \ref{chap:performance}.

\section{Who Does the Testing?}
\label{sec:who_does_the_testing}

Testing that is done by the implementer or someone else with access to the
source code is sometimes called white box testing. (The term is a weak
analogy to black box testing, where the tester does not know how the
component is implemented; "clear box" might be more evocative (恰当).) It
is important to test your own code: don't assume that some testing
organization or user will find things for you. But it's easy to delude
(蛊惑) yourself about how carefully you are testing, so try to ignore the
code and think of hard cases, not easy ones. To quote Don Knuth describing
how he creates tests for the TEX formatter, "I get into the meanest
(低劣的), nastiest (污秽的) frame of mind that I can manage, and I write
the nastiest [testing] code I can think of; then I turn around and embed
(嵌入) that in even nastier constructions that are almost obscene
(可憎的)." The reason for testing is to find bugs, not to declare the
program working. Therefore the tests should be tough, and when they find
problems, that is a vindication (辩护) of your methods, not a cause for
alarm.

Black box testing means that the tester has no knowledge of or access to
the innards (内脏) of the code. It finds different kinds of errors, because
the tester has different assumptions about where to look. Boundary
conditions are a good place to begin black box testing; high-volume,
perverse (错误的), and illegal inputs are good follow-ons (后续). Of course
you should also test the ordinary "middle of the road" or conventional uses
of the program to verify basic functionality.

Real users are the next step. New users find new bugs, because they probe
the program in unexpected ways. It is important to do this kind of testing
before the program is released to the world though, sadly, many programs
are shipped without enough testing of any kind. Beta releases of software
are an attempt to have numerous real users test a program before it is
finalized, but beta releases should not be used as a substitute for
thorough testing. As software systems get larger and more complex, and
development schedules get shorter, however, the pressure to ship without
adequate testing increases.

It's hard to test interactive programs, especially if they involve mouse
input.  Some testing can be done by scripts (whose properties depend on
language, environment, and the like). Interactive programs should be
controllable from scripts that simulate user behaviors so they can be
tested by programs. One technique is to capture the actions of real users
and replay them; another is to create a scripting language that describes
sequences and timing of events.

Finally, give some thought to how to test the tests themselves. We
mentioned in Chapter \ref{chap:debug} the confusion caused by a faulty test
program for a list package.  A regression suite infected by an error will
cause trouble for the rest of time.  The results of a set of tests will not
mean much if the tests themselves are flawed (有缺陷的).

\section{Testing the Markov Program}
\label{sec:testing_the_markov_program}

The Markov program of Chapter \ref{chap:desipl} is sufficiently intricate
(复杂) that it needs careful testing. It produces nonsense, which is hard
to analyze for validity, and we wrote multiple versions in several
languages. As a final complication, its output is random and different each
time. How can we apply some of the lessons of this chapter to testing this
program?

The first set of tests consists of a handful of tiny files that check
boundary conditions, to make sure the program produces the right output for
inputs that contain only a few words. For prefixes of length two, we use
five files that contain respectively (with one word per line)
% TODO: The "(empty file)" should be italic.
\begin{wellcode}
    (empty file)
    a
    a b
    a b c
    a b c d
\end{wellcode}
For each file, the output should be identical to the input. These checks
uncovered several off-by-one (单一断接) errors in initializing the table
and starting and stopping the generator.

A second test verified conservation (保存) properties. For two-word
prefixes, every word, every pair, and every triple that appears in the
output of a run must occur in the input as well. We wrote an Awk program
that reads the original input into a giant array, builds arrays of all
pairs and triples, then reads the Markov output into another array and
compares the two:
\begin{wellcode}
    # markov test: check that all words, pairs, triples in
    # output ARGV[2] are in original input ARGV[1]
    BEGIN {
        while (getline <ARGV[1] > 0)
            for (i = 1; i <= NF; i++) {
                wd[++nw] = $i   # input words
                single[$i]++
            }
        for (i = 1; i < nw; i++)
            pair[wd[i],wd[i+1]]++
        for (i = 1; i < nw-1; i++)
            triple[wd[i],wd[i+1],wd[i+2]]++

        while (getline <ARGV[2] > 0) {
            outwd[++ow] = $0    # output words
            if (!($0 in single))
                print "unexpected word", $0
        }
        for (i = 1; i < ow; i++)
            if (!((outwd[i],outwd[i+1]) in pair))
                print "unexpected pair", outwd[i], outwd[i+1]
        for (i = 1; i < ow-1; i++)
            if (!((outwd[i],outwd[i+1],outwd[i+2]) in triple))
                print "unexpected triple",
                    outwd[i], outwd[i+1], outwd[i+2]
    }
\end{wellcode}
We made no attempt to build an efficient test, just to make the test
program as simple as possible. It takes six or seven seconds to check a
10,000 word output file against a 42,685 word input file, not much longer
than some versions of Markov take to generate it. Checking conservation
caught a major error in our Java implementation: the program sometimes
overwrote (重写) hash table entries because it used references instead of
making copies of prefixes.

This test illustrates the principle that it can be much easier to verify a
property of the output than to create the output itself. For instance it is
easier to check that a file is sorted than to sort it in the first place.

A third test is statistical in nature. The input consists of the sequence
\begin{wellcode}
    a b a b c ... a b d ...
\end{wellcode}
with ten occurrences of \verb'abc' for each \verb'abd'. The output should
have about 10 times as many \texttt{c}'s as \texttt{d}'s if the random
selection is working properly. We confirm this with \verb'freq', of course.

The statistical test showed that an early version of the Java program,
which associated counters with each suffix, produced 20 \texttt{c}'s for
every \texttt{d}, twice as many as it should have. After some head
scratching, we realized that Java's random number generator returns
negative as well as positive integers; the factor of two occurred because
the range of values was twice as large as expected, so twice as many values
would be zero modulo the counter; this favored the first element in the
list, which happened to be \verb'c'.  The fix was to take the absolute
% FIXME: The hyperlink of footnote doesn't response when click.
value before the modulus\footnote{Unrecognized word.}. Without this test,
we would never have discovered the error; to the eye, the output looked
fine.

Finally, we gave the Markov program plain English text to see that it
produced beautiful nonsense. Of course, we also ran this test early in the
development of the program. But we didn't stop testing when the program
handled regular input, because nasty cases will come up in practice.
Getting the easy cases right is seductive (富有魅力的); hard cases must be
tested too. Automated, systematic testing is the best way to avoid this
trap.

All of the testing was mechanized. A shell script generated necessary input
data, ran and timed the tests, and printed any anomalous output. The script
was configurable so the same tests could be applied to any version of
Markov, and every time we made a set of changes to one of the programs, we
ran all the tests again to make sure that nothing was broken.
