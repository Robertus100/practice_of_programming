% vim: ts=4 sts=4 sw=4 et tw=75
\chapter{Interfaces}
\label{chap:interface}
\begin{quote}
    Before I build a wall I'd ask to know \\
    What I was walling in or walling out, \\
    And to whom I was like to give offense. \\
    Something there is that doesn't love a wall. \\
    That wants it down.
\end{quote}
\begin{quotesrc}
    Robert Frost, \bookname{Mending Wall}
\end{quotesrc}

The essence (精髓) of design is to balance competing goals and constraints.
Although there may be many tradeoffs when one is writing a small
self-contained system, the ramifications (分叉) of particular choices
remain within the system and affect only the individual programmer. But
when code is to be used by others, decisions have wider repercussions
(反响).

Among the issues to be worked out in a design are
\begin{itemize}
    \item Interfaces: what services and access are provided? The interface
        is in effect a contract between supplier and customer. The desire
        is to provide services that are uniform and convenient, with enough
        functionality to be easy to use but not so much as to become
        unwieldy (笨拙).
    \item Information hiding: what information is visible and what is
        private? An interface must provide straightforward access to the
        components while hiding details of the implementation so they can
        be changed without affecting users.
    \item Resource management: who is responsible for managing memory and
        other limited resources? Here, the main problems are allocating and
        freeing storage, and managing shared copies of information.
    \item Error handling: who detects errors, who reports them, and how?
        When an error is detected, what recovery is attempted?
\end{itemize}

In Chapter \ref{chap:alds} we looked at the individual pieces -- the data
structures -- from which a system is built. In Chapter \ref{chap:desipl},
we looked at how to combine those into a small program. The topic now turns
to the interfaces between components that might come from different
sources. In this chapter we illustrate interface design by building a
library of functions and data structures for a common task. Along the way,
we will present some principles of design. Typically there are an enormous
number of decisions to be made, but most are made almost unconsciously.
Without these principles, the result is often the sort of haphazard
(无计划的) interfaces that frustrate and impede (妨碍) programmers every
day.

\section{Comma-Separated Values}
\label{sec:comma_separated_values}

\emph{Comma-separated values}, or \emph{CSV}, is the term for a natural and
widely used representation for tabular (表格式的) data. Each row of a table
is a line of text; the fields on each line are separated by commas. The
table at the end of the previous chapter might begin this way in CSV
format: \\
\indent\indent ,"250MHz","400MHz","Line of" \\
\indent\indent ,"R10000","Pentium II","source code" \\
\indent\indent C,0.36 sec,0.30 sec,150 \\
\indent\indent Java,4.9,9.2,105

This format is read and written by programs such as spreadsheets; not
coincidentally (巧合), it also appears on web pages for services such as
stock price quotations. A popular web page for stock quotes presents a
display like this:
\begin{figure}[h]
    \centering
\begin{tabular}{|c|c|c|c|c|c|}
    \hline
    \textbf{Symbol}  & \multicolumn{2}{|c|}{\textbf{Last Trade}} &
    \multicolumn{2}{|c|}{\textbf{Change}} & \textbf{Volume}   \\
    \hline
    LU  & 2:19PM    & 86-114    & +4-1/16   & +4.94\%   & 5,804,800 \\
    \hline
    T   & 2:19PM    & 60-11/16  & -1-3/16   & -1.92\%   & 2,468,000 \\
    \hline
    MSFT& 2:24PM    & 106-9/16  & +1-3/8    & +1.31\%   & 11,474,900\\
    \hline
\end{tabular}
\\
\vspace{1em}
{\centering\underline{Download spreadsheet Format}}
\end{figure}

Retrieving numbers by interacting with a web browser is effective but
time-consuming. It's a nuisance (讨厌的) to invoke a browser, wait, watch a
barrage (弹幕) of advertisements, type a list of stocks, wait, wait, wait,
then watch another barrage, all to get a few numbers. To process the
%numbers further requires even more interaction; selecting the "Download
Spreadsheet Format" link retrieves a file that contains much the same
(几乎相同) information in lines of CSV data like these (edited to fit):
\begin{wellcode}
    "LU",86.25,"11/4/1998","2:19PM",+4.0625,
            83.9375,86.875,83.625,5804800
    "T",60.6875,"11/4/1998","2:19PM",-1.1875,
            62.375,62.625,60.4375,2468000
    "MSFT",106.5625,"11/4/1998","2:24PM",+1.375,
            105.8125,107.3125,105.5625,11474900
\end{wellcode}

Conspicuous (显著的) by its absence in this process is the principle of
letting the machine do the work. Browsers let your computer access data on
a remote server, but it would be more convenient to retrieve the data
without forced interaction. Underneath (在..下面) all the button-pushing is
a purely textual procedure -- the browser reads some HTML, you type some
text, the browser sends that to a server and reads some HTML back. With the
right tools and language, it's easy to retrieve the information
automatically.  Here's a program in the language Tcl to access the stock
quote web site and retrieve CSV data in the format above, preceded (开始)
by a few header lines:
\begin{wellcode}
    # getquotes.tcl: stock prices for Lucent, AT&T, Microsoft
    set so [socket quote.yahoo.com 80]  ;# connect to server
    set q "/d/quotes.csv?s=LU+T+MSFT&f=sl1d1t1ohgv"
    puts $so "GET $q HTTP/1.0\r\n\r\n"  ;# send request
    flush $so
    puts [read $so]                     ;# read & print reply
\end{wellcode}
The cryptic sequence \verb'f=...' that follows the ticker symbols is an
undocumented control string, analogous to the first argument of
\verb'printf', that determines what values to retrieve. By experiment, we
determined that \verb's' identifies the stock symbol, \verb'l1' the last
price, \verb'c1' the change since yesterday, and so on. What's important
isn't the details, which are subject to (可以) change anyway, but the
possibility of automation: retrieving the desired information and
converting it into the form we need without any human intervention. We can
let the machine do the work.

It typically takes a fraction of a second to run \verb'getquotes' , far
less than interacting with a browser. Once we have the data, we will want
to process it further, Data formats like CSV work best if there are
convenient libraries for converting to and from the format, perhaps allied
with (联合) some auxiliary processing such as numerical conversions. But we
do not know of an existing public library to handle CSV, so we will write
one ourselves.

In the next few sections, we will build three versions of a library to read
CSV data and convert it into an internal representation. Along the way,
we'll talk about issues that arise when designing software that must work
with other software. For example, there does not appear to be a standard
definition of CSV, so the implementation cannot be based on a precise
specification, a common situation in the design of interfaces.

\section{A Prototype Library}
\label{sec:a_prototype_library}

We are unlikely to get the design of a library or interface right on the
first attempt.  As Fred Brooks once wrote, "plan to throw one away; you
will, anyhow." Brooks was writing about large systems but the idea is
relevant for any substantial (有内容的) piece of software. It's not usually
until you've built and used a version of the program that you understand
the issues well enough to get the design right.

In this spirit, we will approach (接近) the construction of a library for
CSV by building one to throw away, a prototype. Our first version will
ignore many of the difficulties of a thoroughly engineered library, but
will be complete enough to be useful and to let us gain some familiarity
with the problem.

Our starting point is a function \verb'csvgetline' that reads one line of
CSV data from a file into a buffer, splits it into fields in an array,
removes quotes, and returns the number of fields. Over the years, we have
written similar code in almost every language we know, so it's a familiar
task. Here is a prototype version in C; we've marked it as questionable
because it is just a prototype:
\begin{badcode}
    char    buf[200];   /* input line buffer */
    char    *field[20]; /* fields */

    /* csvgetline: read and parse line, return field count */
    /* sample input: "LU",86.25,"11/4/1998","2:19PM",+4.0625 */
    int csvgetline(FILE *fin)
    {
        int     nfield;
        char    *p, *q;

        if (fgets(buf, sizeof(buf), fin) == NULL)
            return -1;
        nfield = 0;
        for (q = buf; (p=strtok(q, ",\n\r")) != NULL; q = NULL)
            field[nfield++] = unquote(p);
        return nfield;
    }
\end{badcode}
The comment at the top of the function includes an example of the input
format that the program accepts; such comments are helpful for programs
that parse messy (散乱的) input.

The CSV format is too complicated to be parsed easily by \verb'scanf' so we
use the C standard library function \verb'strtok'. Each call of
\verb'strtok(p, s)' returns a pointer to the first token within \verb'p'
consisting of characters not in \verb's'; \verb'strtok' terminates the
token by overwriting the following character of the original string with a
null byte. On the first call, \verb'strtok''s first argument is the string
to scan; subsequent calls use \verb'NULL' to indicate that scanning should
resume where it left off (停止) in the previous call.  This is a poor
interface. Because \verb'strtok' stores a variable in a secret place
between calls, only one sequence of calls may be active at one time;
unrelated interleaved (交错的) calls will interfere with each other.

Our function \verb'unquote' removes the leading and trailing quotes that
appear in the sample input above. It does not handle nested quotes,
however, so although sufficient for a prototype, it's not general.
\begin{badcode}
    /* unquote: remove leading and trailing quote */
    char *unquote(char *p)
    {
        if (p[0] == '"') {
            if (p[strlen(p)-1] == '"')
                p[strlen(p)-1] = '\0';
            p++;
        }
        return p;
    }
\end{badcode}

A simple test program helps verify that \verb'csvgetline' works:
\begin{badcode}
    /* csvtest main: test csvgetline function */
    int main(void)
    {
        int i, nf;

        while ((nf = csvgetline(stdin)) != -1)
            for (i = 0; i < nf; i++)
                printf("field[%d] = '%s'\n", i, field[i]);;
            return 0;
    }
\end{badcode}
The \verb'printf' encloses the fields in matching single quotes, which
demarcate (区分) them and
help to reveal bugs that handle white space incorrectly.

We can now run this on the output produced by \verb'getquotes.tcl':
\begin{wellcode}
    % getquotes.tcl | csvtest
    ...
    field[0] = 'LU'
    field[1] = '86.375'
    field[2] = '11/5/1998'
    field[3] = '1:01PM'
    field[4] = '-0.125'
    field[5] = '86'
    field[6] = '86.375'
    field[7] = '85.0625'
    field[8] = '2888600'
    field[0] = 'T'
    field[1] = '61.0625'
    ...
\end{wellcode}
(We have edited out the HTTP header lines.)

Now we have a prototype that seems to work on data of the sort we showed
above.  But it might be prudent (谨慎的) to try it on something else as
well, especially if we plan to let others use it. We found another web site
that downloads stock quotes and obtained a file of similar information but
in a different form: carriage returns (\verb'\r') rather than newlines to
separate records, and no terminating carriage return at the end of the
file.  We've edited and formatted it to fit on the page:
\begin{wellcode}
    "Ticker","Price","Change","Open","Prev Close","Day High",
        "Day Low","52 Week High","52 Week Low","Dividend",
        "Yield","Volume","Average Volume","P/E"
    "LU",86.313,-0.188,86.000,86.500,86.438,85.063,108.50,
        36.18,0.16,0.1,2946700,9675000,N/A
    "T",61,125,0.938,60.375,60.188,61.125,60.000,68.50,
        46.50,1.32,2.1,3061000,4777000,17.0
    "MSFT",107.000,1.500,105.313,105.500,107.188,105.250,
        119.62,59.00,N/A,N/A,7977300,16965000,51.0
\end{wellcode}
With this input, our prototype failed miserably (非常不幸地).

We designed our prototype after examining one data source, and we tested it
originally only on data from that same source. Thus we shouldn't be
surprised when the first encounter with a different source reveals gross
(恶劣的) failings. Long input lines, many fields, and unexpected or missing
separators all cause trouble. This fragile prototype might serve for
personal use or to demonstrate the feasibility of an approach, but no more
than that. It's time to rethink the design before we try another
implementation.

We made a large number of decisions, both implicit and explicit, in the
prototype. Here are some of the choices that were made, not always in the
best way for a general-purpose library. Each raises an issue that needs
more careful attention.
\begin{itemize}
    \item The prototype doesn't handle long input lines or lots of fields.
        It can give wrong answers or crash because it doesn't even check
        for overflows, let alone return sensible values in case of errors.
    \item The input is assumed to consist of lines terminated by newlines.
    \item Fields are separated by commas and surrounding quotes are
        removed. There is no provision (准备) for embedded quotes or
        commas.  The input line is not preserved; it is overwritten by the
        process of creating fields.
    \item No data is saved from one input line to the next: if something is
        to be remembered, a copy must be made.
    \item Access to the fields is through a global variable, the
        \verb'field' array, which is shared by \verb'csvgetline' and
        functions that call it; there is no control over access to the
        \verb'field' contents or the pointers. There is also no attempt to
        prevent access beyond the last field.
    \item The global variables make the design unsuitable for a
        multi-threaded environment or even for two sequences of interleaved
        calls.
    \item The caller must open and close files explicitly;
        \verb'csvgetline' reads only from open files.
    \item Input and splitting are inextricably (无法避免地) linked: each
        call reads a line and splits it into fields, regardless of whether
        the application needs that service.
    \item The return value is the number of fields on the line; each line
        must be split to compute this value. There is also no way to
        distinguish errors from end of file.
    \item There is no way to change any of these properties without
        changing the code.
\end{itemize}

This long yet incomplete list illustrates some of the possible design
tradeoffs.  Each decision is woven through the code. That's fine for a
quick job, like parsing one fixed format from a known source. But what if
the format changes, or a comma appears within a quoted string, or the
server produces a long line or a lot of fields?

It may seem easy to cope (应付), since the "library" is small and only a
prototype anyway. Imagine, however, that after sitting on the shelf
(被搁置的) for a few months or years the code becomes part of a larger
program whose specification changes over time. How will \verb'csvgetline'
adapt? If that program is used by others, the quick choices made in the
original design may spell (招致) trouble that surfaces years later. This
scenario (情节) is representative of the history of many bad interfaces. It
is a sad fact that a lot of quick and dirty code ends up in widely-used
software, where it remains dirty and often not as quick as it should have
been anyway.

\section{A Library for Others}
\label{sec:a_library_for_others}

Using what we learned from the prototype, we now want to build a library
worthy of general use. The most obvious requirement is that we must make
\verb'csvgetline' more robust so it will handle long lines or many fields;
it must also be more careful in the parsing of fields.

To create an interface that others can use, we must consider the issues
listed at the beginning of this chapter: interfaces, information hiding,
resource management, and error handling. The interplay (互相作用) among
these strongly affects the design. Our separation of these issues is a bit
arbitrary, since they are interrelated.

\emph{Interface.}We decided on three basic operations: \\
\indent\indent\texttt{char *csvgetline(FILE *)}: read a new CSV line \\
\indent\indent\texttt{char *csvfield(int n)}: return the \verb'n'-th field
of the current line \\
\indent\indent\texttt{int csvnfield(void)}: return the number of fields on
the current

What function value should \verb'csvgetline' return? It is desirable to
return as much useful information as convenient, which suggests returning
the number of fields, as in the prototype. But then the number of fields
must be computed even if the fields aren't being used. Another possible
value is the input line length, which is affected by whether the trailing
newline is preserved. After several experiments, we decided that
\verb'csvgetline' will return a pointer to the original line of input, or
\verb'NULL' if end of file has been reached.

We will remove the newline at the end of the line returned by
\verb'csvgetline', since it can easily be restored if necessary.

The definition of a field is complicated; we have tried to match what we
observe empirically (经验的) in spreadsheets and other programs. A field is
a sequence of zero or more characters. Fields are separated by commas.
Leading and trailing blanks are preserved. A field may be enclosed in
double-quote characters, in which case it may contain commas. A quoted
field may contain double-quote characters, which are represented by a
doubled double-quote; the CSV field \verb'"x""y"' defines the string
\verb'x"y'.  Fields may be empty; a field specified as "" is empty, and
identical to one specified by adjacent commas.

Fields are numbered from zero. What if the user asks for a non-existent
field by calling \verb'csvfield(-1)' or \verb'csvfield(100000)'? We could
return \verb'""' (the empty string) because this can be printed and
compared; programs that process variable numbers of fields would not have
to take special precautions to deal with non-existent ones. But that choice
provides no way to distinguish empty from non-existent. A second choice
would be to print an error message or even abort; we will discuss shortly
why this is not desirable. We decided to return \verb'NULL', the
conventional value for a non-existent string in C.

\emph{Information hiding.} The library will impose no limits on input line
length or number of fields. To achieve this, either the caller must provide
the memory or the callee (the library) must allocate it. The caller of the
library function \verb'fgets' passes in an array and a maximum size. If the
line is longer than the buffer, it is broken into pieces.  This behavior is
unsatisfactory for the CSV interface, so our library will allocate memory
as it discovers that more is needed.

Thus only \verb'csvgetline' knows about memory management; nothing about
the way that it organizes memory is accessible from outside. The best way
to provide that isolation is through a function interface:
\verb'csvgetline' reads the next line, no matter how big,
\verb'csvfield(n)' returns a pointer to the bytes of the \verb'n'-th field
of the current line, and \verb'csvnfield' returns the number of fields on
the current line.

We will have to grow memory as longer lines or more fields arrive. Details
of how that is done are hidden in the csv functions; no other part of the
program knows how this works, for instance whether the library uses small
arrays that grow, or very large arrays, or something completely different.
Nor does the interface reveal when memory is freed.

If the user calls only \verb'csvgetline', there's no need to split into
fields; lines can be split on demand. Whether field-splitting is eager
(done right away when the line is read) or lazy (done only when a field or
count is needed) or very lazy (only the requested field is split) is
another implementation detail hidden from the user.

\emph{Resource management.} We must decide who is responsible for shared
information.  Does \verb'csvgetline' return the original data or make a
copy? We decided that the return value of \verb'csvgetline' is a pointer to
the original input, which will be overwritten when the next line is read.
Fields will be built in a copy of the input line, and \verb'csvfield' will
return a pointer to the field within the copy. With this arrangement, the
user must make another copy if a particular line or field is to be saved or
changed, and it is the user's responsibility to release that storage when
it is no longer needed.

Who opens and closes the input file? Whoever opens an input file should do
the corresponding close: matching tasks should be done at the same level or
place. We will assume that \verb'csvgetline' is called with a \verb'FILE'
pointer to an already-open file that the caller will close when processing
is complete.

Managing the resources shared or passed across the boundary between a
library and its callers is a difficult task, and there are often sound
(合理的) but conflicting reasons to prefer various design choices. Errors
and misunderstandings about the shared responsibilities are a frequent
source of bugs.

\emph{Error handling.} Because \verb'csvgetline' returns \verb'NULL', there
is no good way to distinguish end of file from an error like running out of
memory; similarly, access to a non-existent field causes no error. By
analogy with \verb'ferror', we could add another function
\verb'csvgeterror' to the interface to report the most recent error, but
for simplicity we will leave it out of this version.

As a principle, library routines should not just die when an error occurs;
error status should be returned to the caller for appropriate action. Nor
should they print messages or pop up dialog boxes, since they may be
running in an environment where a message would interfere with something
else. Error handling is a topic worth a separate discussion of its own,
later in this chapter.

\emph{Specification.} The choices made above should be collected in one
place as a specification of the services that \verb'csvgetline' provides
and how it is to be used.  In a large project, the specification precedes
the implementation, because specifiers and implementers are usually
different people and may be in different organizations.  In practice,
however, work often proceeds in parallel, with specification and code
evolving together, although sometimes the "specification" is written only
after the fact to describe approximately what the code does.

The best approach is to write the specification early and revise it as we
learn from the ongoing implementation. The more accurate and careful a
specification is, the more likely that the resulting program will work
well. Even for personal programs, it is valuable to prepare a reasonably
thorough specification because it encourages consideration of alternatives
and records the choices made.

For our purposes, the specification would include function prototypes and a
detailed prescription (命令) of behavior, responsibilities and assumptions:
\begin{myitemize}
\item Fields are separated by commas. \\
    A field may be enclosed in double-quote characters "...". \\
    A quoted field may contain commas but not newlines. \\
    A quoted field may contain double-quote characters ", represented by
    "".   \\
    Fields may be empty; "" and an empty string both represent an empty
    field. \\
    Leading and trailing white space is preserved.
\item \verb'char *csvgetline(FILE *f)';
    \begin{myitemize}
\item reads one line from open input file; assumes that input lines
    are terminated by \verb'\r', \verb'\n', \verb'\r\n', or \verb'EOF'.
\item returns pointer to line, with terminator removed, or \verb'NULL' if
    \verb'EOF' occurred.
\item line may be of arbitrary length; returns \verb'NULL' if memory
    limit exceeded.
\item line must be treated as read-only storage; caller must make a copy
    to preserve or change contents.
\end{myitemize}
\item \verb'char *csvfield(int n)';
\begin{myitemize}
        \item fields are numbered from 0.
        \item returns \verb'n'-th field from last line read by
            \verb'csvgetline'; returns \verb'NULL' if \verb'n < 0' or
            beyond last field.
        \item fields are separated by commas.
        \item fields may be surrounded by \verb'"..."'; such quotes are
            removed; within \verb'"..."', \verb'""' is replaced by \verb'"'
            and comma is not a separated.
        \item in unquoted fields, quotes are regular characters.
        \item there can be an arbitrary number of fields of any length;
            returns \verb'NULL' if memory limit exceeded.
        \item field must be treated as read-only storage; caller must make
            a copy to preserve or change contents.
        \item behavior undefined if called before \verb'csvgetline' is
            called.
\end{myitemize}
\item \verb'int csvnfield(void)';
\begin{myitemize}
        \item returns number of fields on last line read by
        \verb'csvgetlien'.
        \item behavior undefined if called before \verb'csvgetline' is
            called.
\end{myitemize}
\end{myitemize}

This specification still leaves open questions. For example, what values
should be returned by \verb'csvfield' and \verb'csvnfield' if they are
called after \verb'csvgetline' has encountered \verb'EOF'? How should
ill-formed fields be handled? Nailing down all such puzzles is difficult
even for a tiny system, and very challenging for a large one, though it is
important to try. One often doesn't discover oversights (失察) and
omissions (遗漏) until implementation is underway.

The rest of this section contains a new implementation of \verb'csvgetline'
that matches the specification. The library is broken into two files, a
header \verb'csv.h' that contains the function declarations that represent
the public part of the interface, and an implementation file \verb'csv.c'
that contains the code. Users include \verb'csv.h' in their source code and
link their compiled code with the compiled version of \verb'csv.c'; the
source need never be visible.

Here is the header file:
\begin{wellcode}
    /* csv.h: interface for csv library */
    extern char *csvgetline(FILE *f);   /* read next input line */
    extern char *csvfield(int n);       /* return field n */
    extern int  csvnfield(void);        /* return number of fields */
\end{wellcode}

The internal variables that store text and the internal functions like
\verb'split' are declared \verb'static' so they are visible only within the
file that contains them.  This is the simplest way to hide information in a
C program.
\begin{wellcode}
    enum { NOMEM = -2 };            /* out of memory signal */

    static char *line       = NULL; /* input chars */
    static char *sline      = NULL; /* line copy used by split */
    static int  maxline     = 0;    /* size of line[] and sline[] */
    static char **field     = NULL; /* field pointers */
    static int  maxfield    = 0;    /* size of field[] */
    static int  nfield      = 0;    /* number of fields in field[] */

    static char fieldsep[]  = ",";  /* field separator chars */
\end{wellcode}
The variables are initialized statically as well. These initial values are
used to test whether to create or grow arrays.

These declarations describe a simple data structure. The \verb'line' array
holds the input line; the \verb'sline' array is created by copying
characters from \verb'line' and terminating each field. The \verb'field'
array points to entries in \verb'sline'. This diagram shows the state of
these three arrays after the input line \verb'ab,"cd","e""f",,"g,h"' has
been processed. Shaded elements in \verb'sline' are not part of any field.
\\
% TODO: "shaded elements" mentioned in above.
\begin{center}
    \begin{tikzpicture}
        \pgfmathsetmacro\width{0.5};
        \pgfmathsetmacro\height{1.0};
        \pgfmathsetmacro\dist{1.5};

        \node at (0, \height/2) {\texttt{field}};
        \node at (0, \height/2 + \dist) {\texttt{sline}};
        \node at (0, \height/2 + 2*\dist) {\texttt{line}};

        \draw (\dist, \dist) -- (\dist + 21*\width, \dist);
        \draw (\dist, \dist + \height) -- (\dist + 21*\width,
        \dist + \height);
        \draw (\dist, 2*\dist) -- (\dist + 21*\width, 2*\dist);
        \draw (\dist, 2*\dist + \height) -- (\dist + 21*\width,
        2*\dist + \height);

        \newcounter{i};
        \setcounter{i}{0};
        \foreach \ind/\data in { 0/\texttt{a}, 1/\texttt{b},
            2/\texttt{,}, 3/\texttt{"}, 4/\texttt{c},
            5/\texttt{d}, 6/\texttt{"}, 7/\texttt{,}, 8/\texttt{"},
            9/\texttt{e}, 10/\texttt{"}, 11/\texttt{"}, 12/\texttt{f},
            13/\texttt{"}, 14/\texttt{,}, 15/\texttt{,}, 16/\texttt{"},
            17/\texttt{g}, 18/\texttt{,}, 19/\texttt{h}, 20/\texttt{"}
        } {

            \node at (\arabic{i} * \width + \dist + \width / 2, 2 * \dist +
            \height / 2) {\data};
            \draw (\arabic{i} * \width + \dist, 2*\dist) --
            (\arabic{i} * \width + \dist, 2 * \dist + \height);
            \addtocounter{i}{1};
        }
        \draw (\arabic{i} * \width + \dist, 2*\dist) --
        (\arabic{i} * \width + \dist, 2 * \dist + \height);

        \setcounter{i}{0};
        \foreach \ind/\data in { 0/\texttt{a}, 1/\texttt{b},
            2/\texttt{\textbackslash0}, 3/\texttt{"}, 4/\texttt{c},
            5/\texttt{d}, 6/\texttt{\textbackslash0},
            7/\texttt{\textbackslash0}, 8/\texttt{"}, 9/\texttt{e},
            10/\texttt{"}, 11/\texttt{f}, 12/\texttt{\textbackslash0},
            13/\texttt{"}, 14/\texttt{\textbackslash0},
            15/\texttt{\textbackslash0}, 16/\texttt{"}, 17/\texttt{g},
            18/\texttt{,}, 19/\texttt{h}, 20/\texttt{\textbackslash0}
        } {

            \node at (\arabic{i} * \width + \dist + \width / 2, \dist +
            \height / 2) {\data};
            \draw (\arabic{i} * \width + \dist, \dist) --
            (\arabic{i} * \width + \dist, \dist + \height);
            \addtocounter{i}{1};
        }
        \draw (\arabic{i} * \width + \dist, \dist) --
        (\arabic{i} * \width + \dist, \dist + \height);

        \node at (\width / 2 + \dist, \height / 2) {\texttt{0}};
        \node at (4 * \width + \width / 2 + \dist, \height / 2)
        {\texttt{1}};
        \node at (9 * \width + \width / 2 + \dist, \height / 2)
        {\texttt{2}};
        \node at (15 * \width + \width / 2 + \dist, \height / 2)
        {\texttt{3}};
        \node at (17 * \width + \width / 2 + \dist, \height / 2)
        {\texttt{4}};

        \draw[thick, ->] (\width / 2 + \dist, \height / 2 + 0.3) -- (\width
        / 2 + \dist, \dist);

        \draw[thick, ->] (4 * \width + \width / 2 + \dist, \height / 2 +
        0.3) -- (4 * \width + \width / 2 + \dist, \dist);

        \draw[thick, ->] (9 * \width + \width / 2 + \dist, \height / 2 +
        0.3) -- (9 * \width + \width / 2 + \dist, \dist);

        \draw[thick, ->] (15 * \width + \width / 2 + \dist, \height / 2 +
        0.3) -- (15 * \width + \width / 2 + \dist, \dist);

        \draw[thick, ->] (17 * \width + \width / 2 + \dist, \height / 2 +
        0.3) -- (17 * \width + \width / 2 + \dist, \dist);

    \end{tikzpicture}
\end{center}

Here is the function \verb'csvgetline' itself:
\begin{wellcode}
    /* csvgetline: get one line, grow as needed */
    /* sample input: "LU",86,25,"11/4/1998","2:19PM",+4.0625 */
    char *csvgetline(FILE *fin)
    {
        int     i, c;
        char    *newl, *news;

        if (line == NULL) {     /* allocate on first call */
            maxline = maxfield = 1;
            line = (char *)malloc(maxline);
            sline = (char *)malloc(maxline);
            field = (char **)malloc(maxfield * sizeof(field[0]));
            if (line == NULL || sline == NULL || field == NULL) {
                reset();
                return NULL;    /* out of memory */
            }
        }
        for (i = 0; (c=getc(fin)) != EOF && !endofline(fin, c); i++) {
            if (i >= maxline - 1) { /* grow line */
                maxline *= 2;       /* double current size */
                newl = (char *)realloc(line, maxline);
                news = (char *)realloc(sline, maxline);
                if (newl == NULL || news == NULL) {
                    reset();
                    return NULL;    /* out of memory */
                }
                line = newl;
                sline = news;
            }
            line[i] = c;
        }
        line[i] = '\0';
        if (split() == NOMEM) {
            reset();
            return NULL;    /* out of memory */
        }
        return (c == EOF && i == 0) ? NULL : line;
    }
\end{wellcode}
An incoming line is accumulated in \verb'line', which is grown as necessary
by a call to \verb'realloc'; the size is doubled on each growth, as in
Section \ref{sec:growing_arrays}. The \verb'sline' array is kept the same
size as \verb'line'; \verb'csvgetline' calls \verb'split' to create the
field pointers in a separated array \verb'field', which is also grown as
needed.

As is our custom, we start the arrays very small and grow them on demand,
to guarantee that the array-growing code is exercised. If allocation fails,
we call \verb'reset' to restore the globals to their starting state, so a
subsequent call to \verb'csvgetline' has a chance of succeeding:
\begin{wellcode}
    static void reset(void)
    {
        free(line); /* free(NULL) permitted by ANSI C */
        free(sline);
        free(field);
        line = NULL;
        sline = NULL;
        field = NULL;
        maxline = maxfield = nfield = 0;
    }
\end{wellcode}

The \verb'endofline' function handles the problem that an input line may be
terminated by a carriage return, a newline, both, or even \verb'EOF':
\begin{wellcode}
    /* endofline: check for and consume \r, \n, \r\n, or EOF */
    static int endofline(FILE *fin, int c)
    {
        int eol;

        eol = (c == '\r' || c == '\n');
        if (c == '\r') {
            c = getc(fin);
            if (c != '\n' && c != EOF)
                ungetc(c, fin); /* read too far; put c back */
        }
        return eol;
    }
\end{wellcode}
A separate function is necessary, since the standard input functions do not
handle the rich variety of perverse formats encountered in real inputs.

Our prototype used \verb'strtok' to find the next token by searching for a
separator character, normally a comma, but this made it impossible to
handle quoted commas.  A major change in the implementation of \verb'split'
is necessary, though its interface need not change. Consider these input
lines:
\begin{wellcode}
    "",,""
    ,"",
    ,,
\end{wellcode}

Each line has three empty fields. Making sure that \verb'split' parses them
and other odd (古怪的) inputs correctly complicates it significantly, an
example of how special cases and boundary conditions can come to dominate a
program.
\begin{wellcode}
    /* split: split line into fields */
    static int split(void)
    {
        char    *p, **newf;
        char    *sepp;  /* pointer to temporary separator character */
        int     sepc;   /* temporary separator character */

        nfield = 0;
        if (line[0] == '\0')
            return 0;
        strcpy(sline, line);
        p = sline;

        do {
            if (nfield >= maxfield) {
                maxfield *= 2;  /* double current size */
                newf = (char **)realloc(field,
                        maxfield * sizeof(field[0]));
                if (newf == NULL)
                    return NOMEM;
                field = newf;
            }
            if (*p == '"')
                sepp = advquoted(++p);  /* skip initial quote */
            else
                sepp = p + strcspn(p, fieldsep);
            sepc = sepp[0];
            sepp[0] = '\0'; /* terminate field */
            field[nfield++] = p;
            p = sepp + 1;
        } while (sepc == ',');

        return nfield;
    }
\end{wellcode}
The loop grows the array of field pointers if necessary, then calls one of
two other functions to locate and process the next field. If the field
begins with a quote, \verb'advquoted' finds the field and returns a pointer
to the separator that ends the field.  Otherwise, to find the next comma we
use the library function \verb'strcspn(p, s)', which searches a string
\verb'p' for the next occurrence of any character in string \verb's'; it
returns the number of characters skipped over.

Quotes within a field are represented by two adjacent quotes, so
\verb'advquoted' squeezes those into a single one; it also removes the
quotes that surround the field.  Some complexity is added by an attempt to
cope with plausible (貌似合理的) inputs that don't match the specification,
such as \verb'"abc"def'. In such cases, we append whatever follows the
second quote until the next separator as part of this field.  Microsoft
Excel appears to use a similar algorithm.
\begin{wellcode}
    /* advquoted: quoted field; return pointer to next separator */
    static char *advquoted(char *p)
    {
        int i, j;

        for (i = j = 0; p[j] != '\0'; i++, j++) {
            if (p[j] == '"' && p[++j] != '"') {
                /* copy up to next separator or \0 */
                int k = strcspn(p+j, fieldsep);
                memmove(p+i, p+j, k);
                i += k;
                j += k;
                break;
            }
            p[i] = p[j];
        }
        p[i] = '\0';
        return p + j;
    }
\end{wellcode}

Since the input line is already split, \verb'csvfield' and \verb'csvnfield'
are trivial (琐细的).
\begin{wellcode}
    /* csvfield: return pointer to n-th field */
    char *csvfield(int n)
    {
        if (n < 0 || n >= nfield)
            return NULL;
        return field[n];
    }

    /* csvnfield: return number of fields */
    int csvnfield(void)
    {
        return nfield;
    }
\end{wellcode}

Finally, we can modify the test driver to exercise this version of the
library; since it keeps a copy of the input line, which the prototype does
not, it can print the original line before printing the fields:
\begin{wellcode}
    /* csvtest main: test CSV library */
    int main(void)
    {
        int     i;
        char    *line;

        while ((line = csvgetline(stdin)) != NULL) {
            printf("line = '%s'\n", line);
            for (i = 0; i < csvnfield(); i++)
                printf("field[%d] = '%s'\n", i, csvfield(i));
        }
        return 0;
    }
\end{wellcode}

This completes our C version. It handles arbitrarily large inputs and does
something sensible even with perverse (歪曲的) data. The price is that it
is more than four times as long as the first prototype and some of the code
is intricate (复杂的). Such expansion of size and complexity is a typical
result of moving from prototype to production.

\begin{exercise}
    There are several degrees of laziness for field-splitting; among the
    possibilities are to split all at once but only when some field is
    requested, to split only the field requested, or to split up to the
    field requested. Enumerate possibilities, assess their potential
    difficulty and benefits, then write them and measure their speeds.
\end{exercise}

\begin{exercise}
    Add a facility so separators can be changed (a) to an arbitrary class
    of characters; (b) to different separators for different fields; (c) to
    a regular expression (see Chapter \ref{chap:notation}). What should the
    interface look like?
\end{exercise}

\begin{exercise}
    We chose to use the static initialization provided by C as the basis of
    a one-time switch: if a pointer is \verb'NULL' on entry, initialization
    is performed. Another possibility is to require the user to call an
    explicit initialization function, which could include suggested initial
    sizes for arrays. Implement a version that combines the best of both.
    What is the role of \verb'reset' in your implementation ?
\end{exercise}

\begin{exercise}
    Design and implement a library for creating CSV-formatted data. The
    simplest version might take an array of strings and print them with
    quotes and commas. A more sophisticated version might use a format
    string analogous to \verb'printf'.  Look at Chapter \ref{chap:notation}
    for some suggestions on notation.
\end{exercise}
