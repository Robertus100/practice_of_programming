% vim: ts=4 sts=4 sw=4 et tw=75
\chapter{Interfaces}
\label{chap:interface}
\begin{quote}
    Before I build a wall I'd ask to know \\
    What I was walling in or walling out, \\
    And to whom I was like to give offence. \\
    Somthing there is that doesn't love a wall. \\
    That wants it down.
\end{quote}
\begin{quotesrc}
    Robert Frost, \bookname{Mending Wall}
\end{quotesrc}

The essence (精髓) of design is to balance competing goals and constraints.
Although there may be many tradeoffs when one is writing a small
self-contained system, the ramifications (分叉) of particular choices
remain within the system and affect only the individual programmer. But
when code is to be used by others, decisions have wider repercussions
(反响).

Among the issues to be worked out in a design are
\begin{itemize}
    \item Interfaces: what services and access are provided? The interface
        is in effect a contract between supplier and customer. The desire
        is to provide services that are uniform and convenient, with enough
        functionality to be easy to use but not so much as to become
        unwieldy (笨拙).
    \item Information hiding: what information is visible and what is
        private? An interface must provide straightforward access to the
        components while hiding details of the implementation so they can
        be changed without affecting users.
    \item Resource management: who is responsible for managing memory and
        other limited resources? Here, the main problems are allocating and
        freeing storage, and managing shared copies of information.
    \item Error handling: who detects errors, who reports them, and how?
        When an error is detected, what recovery is attempted?
\end{itemize}

In Chapter \ref{chap:alds} we looked at the individual pieces -- the data
structures -- from which a system is built. In Chapter \ref{chap:desipl},
we looked at how to combine those into a small program. The topic now turns
to the interfaces between components that might come from different
sources. In this chapter we illustrate interface design by building a
library of functions and data structures for a common task. Along the way,
we will present some principles of design. Typically there are an enormous
number of decisions to be made, but most are made almost unconsciously.
Without these principles, the result is often the sort of haphazard
(无计划的) interfaces that frustrate and impede (妨碍) programmers every
day.

\section{Comma-Separated Values}
\label{sec:comma_separated_values}

\emph{Comma-separated values}, or \emph{CSV}, is the term for a natural and
widely used representation for tabular (表格式的) data. Each row of a table
is a line of text; the fields on each line are separated by commas. The
table at the end of the previous chapter might begin this way in CSV
format: \\
\indent\indent ,"250MHz","400MHz","Line of" \\
\indent\indent ,"R10000","Pentium II","source code" \\
\indent\indent C,0.36 sec,0.30 sec,150 \\
\indent\indent Java,4.9,9.2,105

This format is read and written by programs such as spreadsheets; not
coincidentally (巧合), it also appears on web pages for services such as
stock price quotations. A popular web page for stock quotes presents a
display like this:
\begin{figure}[h]
    \centering
\begin{tabular}{|c|c|c|c|c|c|}
    \hline
    \textbf{Symbol}  & \multicolumn{2}{|c|}{\textbf{Last Trade}} &
    \multicolumn{2}{|c|}{\textbf{Change}} & \textbf{Volume}   \\
    \hline
    LU  & 2:19PM    & 86-114    & +4-1/16   & +4.94\%   & 5,804,800 \\
    \hline
    T   & 2:19PM    & 60-11/16  & -1-3/16   & -1.92\%   & 2,468,000 \\
    \hline
    MSFT& 2:24PM    & 106-9/16  & +1-3/8    & +1.31\%   & 11,474,900\\
    \hline
\end{tabular}
\\
\vspace{1em}
{\centering\underline{Download spreadsheet Format}}
\end{figure}

Retrieving numbers by interacting with a web browser is effective but
time-consuming. It's a nuisance (讨厌的) to invoke a browser, wait, watch a
barrage (弹幕) of advertisements, type a list of stocks, wait, wait, wait,
then watch another barrage, all to get a few numbers. To process the
%numbers further requires even more interaction; selecting the "Download
Spreadsheet Format" link retrieves a file that contains much the same
(几乎相同) information in lines of CSV data like these (edited to fit):
\begin{wellcode}
    "LU",86.25,"11/4/1998","2:19PM",+4.0625,
            83.9375,86.875,83.625,5804800
    "T",60.6875,"11/4/1998","2:19PM",-1.1875,
            62.375,62.625,60.4375,2468000
    "MSFT",106.5625,"11/4/1998","2:24PM",+1.375,
            105.8125,107.3125,105.5625,11474900
\end{wellcode}

Conspicuous (显著的) by its absence in this process is the principle of
letting the machine do the work. Browsers let your computer access data on
a remote server, but it would be more convenient to retrieve the data
without forced interaction. Underneath (在..下面) all the button-pushing is
a purely textual procedure -- the browser reads some HTML, you type some
text, the browser sends that to a server and reads some HTML back. With the
right tools and language, it's easy to retrieve the information
automatically.  Here's a program in the language Tcl to access the stock
quote web site and retrieve CSV data in the format above, preceded (开始)
by a few header lines:
\begin{wellcode}
    # getquote.tcl: stock prices for Lucent, AT&T, Microsoft
    set so [socket quote.yahoo.com 80]  ;# connect to server
    set q "/d/quotes.csv?s=LU+T+MSFT&f=sl1d1t1ohgv"
    puts $so "GET $q HTTP/1.0\r\n\r\n"  ;# send request
    flush $so
    puts [read $so]                     ;# read & print reply
\end{wellcode}
The cryptic sequence \verb'f=...' that follows the ticker symbols is an
undocumented control string, analogous to the first argument of
\verb'printf', that determines what values to retrieve. By experiment, we
determined that \verb's' identifies the stock symbol, \verb'l1' the last
price, \verb'c1' the change since yesterday, and so on. What's important
isn't the details, which are subject to (可以) change anyway, but the
possibility of automation: retrieving the desired information and
converting it into the form we need without any human intervention. We can
let the machine do the work.

It typically takes a fraction of a second to run \verb'getquotes' , far
less than interacting with a browser. Once we have the data, we will want
to process it further, Data formats like CSV work best if there are
convenient libraries for converting to and from the format, perhaps allied
with (联合) some auxiliary processing such as numerical conversions. But we
do not know of an existing public library to handle CSV, so we will write
one ourselves.

In the next few sections, we will build three versions of a library to read
CSV data and convert it into an internal representation. Along the way,
we'll talk about issues that arise when designing software that must work
with other software. For example, there does not appear to be a standard
definition of CSV, so the implementation cannot be based on a precise
specification, a common situation in the design of interfaces.
