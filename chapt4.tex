% vim: ts=4 sts=4 sw=4 et tw=75
\chapter{Interfaces}
\label{chap:interface}
\begin{quote}
    Before I build a wall I'd ask to know \\
    What I was walling in or walling out, \\
    And to whom I was like to give offence. \\
    Somthing there is that doesn't love a wall. \\
    That wants it down.
\end{quote}
\begin{quotesrc}
    Robert Frost, \bookname{Mending Wall}
\end{quotesrc}

The essence (精髓) of design is to balance competing goals and constraints.
Although there may be many tradeoffs when one is writing a small
self-contained system, the ramifications (分叉) of particular choices
remain within the system and affect only the individual programmer. But
when code is to be used by others, decisions have wider repercussions
(反响).

Among the issues to be worked out in a design are
\begin{itemize}
    \item Interfaces: what services and access are provided? The interface
        is in effect a contract between supplier and customer. The desire
        is to provide services that are uniform and convenient, with enough
        functionality to be easy to use but not so much as to become
        unwieldy (笨拙).
    \item Information hiding: what information is visible and what is
        private? An interface must provide straightforward access to the
        components while hiding details of the implementation so they can
        be changed without affecting users.
    \item Resource management: who is responsible for managing memory and
        other limited resources? Here, the main problems are allocating and
        freeing storage, and managing shared copies of information.
    \item Error handling: who detects errors, who reports them, and how?
        When an error is detected, what recovery is attempted?
\end{itemize}

In Chapter \ref{chap:alds} we looked at the individual pieces -- the data
structures -- from which a system is built. In Chapter \ref{chap:desipl},
we looked at how to combine those into a small program. The topic now turns
to the interfaces between components that might come from different
sources. In this chapter we illustrate interface design by building a
library of functions and data structures for a common task. Along the way,
we will present some principles of design. Typically there are an enormous
number of decisions to be made, but most are made almost unconsciously.
Without these principles, the result is often the sort of haphazard
(无计划的) interfaces that frustrate and impede (妨碍) programmers every
day.
