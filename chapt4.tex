% vim: ts=4 sts=4 sw=4 et tw=75
\chapter{Interfaces}
\label{chap:interface}
\begin{quote}
    Before I build a wall I'd ask to know \\
    What I was walling in or walling out, \\
    And to whom I was like to give offense. \\
    Something there is that doesn't love a wall. \\
    That wants it down.
\end{quote}
\begin{quotesrc}
    Robert Frost, \bookname{Mending Wall}
\end{quotesrc}

The essence (精髓) of design is to balance competing goals and constraints.
Although there may be many tradeoffs when one is writing a small
self-contained system, the ramifications (分叉) of particular choices
remain within the system and affect only the individual programmer. But
when code is to be used by others, decisions have wider repercussions
(反响).

Among the issues to be worked out in a design are
\begin{itemize}
    \item Interfaces: what services and access are provided? The interface
        is in effect a contract between supplier and customer. The desire
        is to provide services that are uniform and convenient, with enough
        functionality to be easy to use but not so much as to become
        unwieldy (笨拙).
    \item Information hiding: what information is visible and what is
        private? An interface must provide straightforward access to the
        components while hiding details of the implementation so they can
        be changed without affecting users.
    \item Resource management: who is responsible for managing memory and
        other limited resources? Here, the main problems are allocating and
        freeing storage, and managing shared copies of information.
    \item Error handling: who detects errors, who reports them, and how?
        When an error is detected, what recovery is attempted?
\end{itemize}

In Chapter \ref{chap:alds} we looked at the individual pieces -- the data
structures -- from which a system is built. In Chapter \ref{chap:desipl},
we looked at how to combine those into a small program. The topic now turns
to the interfaces between components that might come from different
sources. In this chapter we illustrate interface design by building a
library of functions and data structures for a common task. Along the way,
we will present some principles of design. Typically there are an enormous
number of decisions to be made, but most are made almost unconsciously.
Without these principles, the result is often the sort of haphazard
(无计划的) interfaces that frustrate and impede (妨碍) programmers every
day.

\section{Comma-Separated Values}
\label{sec:comma_separated_values}

\emph{Comma-separated values}, or \emph{CSV}, is the term for a natural and
widely used representation for tabular (表格式的) data. Each row of a table
is a line of text; the fields on each line are separated by commas. The
table at the end of the previous chapter might begin this way in CSV
format: \\
\indent\indent ,"250MHz","400MHz","Line of" \\
\indent\indent ,"R10000","Pentium II","source code" \\
\indent\indent C,0.36 sec,0.30 sec,150 \\
\indent\indent Java,4.9,9.2,105

This format is read and written by programs such as spreadsheets; not
coincidentally (巧合), it also appears on web pages for services such as
stock price quotations. A popular web page for stock quotes presents a
display like this:
\begin{figure}[h]
    \centering
\begin{tabular}{|c|c|c|c|c|c|}
    \hline
    \textbf{Symbol}  & \multicolumn{2}{|c|}{\textbf{Last Trade}} &
    \multicolumn{2}{|c|}{\textbf{Change}} & \textbf{Volume}   \\
    \hline
    LU  & 2:19PM    & 86-114    & +4-1/16   & +4.94\%   & 5,804,800 \\
    \hline
    T   & 2:19PM    & 60-11/16  & -1-3/16   & -1.92\%   & 2,468,000 \\
    \hline
    MSFT& 2:24PM    & 106-9/16  & +1-3/8    & +1.31\%   & 11,474,900\\
    \hline
\end{tabular}
\\
\vspace{1em}
{\centering\underline{Download spreadsheet Format}}
\end{figure}

Retrieving numbers by interacting with a web browser is effective but
time-consuming. It's a nuisance (讨厌的) to invoke a browser, wait, watch a
barrage (弹幕) of advertisements, type a list of stocks, wait, wait, wait,
then watch another barrage, all to get a few numbers. To process the
%numbers further requires even more interaction; selecting the "Download
Spreadsheet Format" link retrieves a file that contains much the same
(几乎相同) information in lines of CSV data like these (edited to fit):
\begin{wellcode}
    "LU",86.25,"11/4/1998","2:19PM",+4.0625,
            83.9375,86.875,83.625,5804800
    "T",60.6875,"11/4/1998","2:19PM",-1.1875,
            62.375,62.625,60.4375,2468000
    "MSFT",106.5625,"11/4/1998","2:24PM",+1.375,
            105.8125,107.3125,105.5625,11474900
\end{wellcode}

Conspicuous (显著的) by its absence in this process is the principle of
letting the machine do the work. Browsers let your computer access data on
a remote server, but it would be more convenient to retrieve the data
without forced interaction. Underneath (在..下面) all the button-pushing is
a purely textual procedure -- the browser reads some HTML, you type some
text, the browser sends that to a server and reads some HTML back. With the
right tools and language, it's easy to retrieve the information
automatically.  Here's a program in the language Tcl to access the stock
quote web site and retrieve CSV data in the format above, preceded (开始)
by a few header lines:
\begin{wellcode}
    # getquotes.tcl: stock prices for Lucent, AT&T, Microsoft
    set so [socket quote.yahoo.com 80]  ;# connect to server
    set q "/d/quotes.csv?s=LU+T+MSFT&f=sl1d1t1ohgv"
    puts $so "GET $q HTTP/1.0\r\n\r\n"  ;# send request
    flush $so
    puts [read $so]                     ;# read & print reply
\end{wellcode}
The cryptic sequence \verb'f=...' that follows the ticker symbols is an
undocumented control string, analogous to the first argument of
\verb'printf', that determines what values to retrieve. By experiment, we
determined that \verb's' identifies the stock symbol, \verb'l1' the last
price, \verb'c1' the change since yesterday, and so on. What's important
isn't the details, which are subject to (可以) change anyway, but the
possibility of automation: retrieving the desired information and
converting it into the form we need without any human intervention. We can
let the machine do the work.

It typically takes a fraction of a second to run \verb'getquotes' , far
less than interacting with a browser. Once we have the data, we will want
to process it further, Data formats like CSV work best if there are
convenient libraries for converting to and from the format, perhaps allied
with (联合) some auxiliary processing such as numerical conversions. But we
do not know of an existing public library to handle CSV, so we will write
one ourselves.

In the next few sections, we will build three versions of a library to read
CSV data and convert it into an internal representation. Along the way,
we'll talk about issues that arise when designing software that must work
with other software. For example, there does not appear to be a standard
definition of CSV, so the implementation cannot be based on a precise
specification, a common situation in the design of interfaces.

\section{A Prototype Library}
\label{sec:a_prototype_library}

We are unlikely to get the design of a library or interface right on the
first attempt.  As Fred Brooks once wrote, "plan to throw one away; you
will, anyhow." Brooks was writing about large systems but the idea is
relevant for any substantial (有内容的) piece of software. It's not usually
until you've built and used a version of the program that you understand
the issues well enough to get the design right.

In this spirit, we will approach (接近) the construction of a library for
CSV by building one to throw away, a prototype. Our first version will
ignore many of the difficulties of a thoroughly engineered library, but
will be complete enough to be useful and to let us gain some familiarity
with the problem.

Our starting point is a function \verb'csvgetline' that reads one line of
CSV data from a file into a buffer, splits it into fields in an array,
removes quotes, and returns the number of fields. Over the years, we have
written similar code in almost every language we know, so it's a familiar
task. Here is a prototype version in C; we've marked it as questionable
because it is just a prototype:
\begin{badcode}
    char    buf[200];   /* input line buffer */
    char    *field[20]; /* fields */

    /* csvgetline: read and parse line, return field count */
    /* sample input: "LU",86.25,"11/4/1998","2:19PM",+4.0625 */
    int csvgetline(FILE *fin)
    {
        int     nfield;
        char    *p, *q;

        if (fgets(buf, sizeof(buf), fin) == NULL)
            return -1;
        nfield = 0;
        for (q = buf; (p=strtok(q, ",\n\r")) != NULL; q = NULL)
            field[nfield++] = unquote(p);
        return nfield;
    }
\end{badcode}
The comment at the top of the function includes an example of the input
format that the program accepts; such comments are helpful for programs
that parse messy (散乱的) input.

The CSV format is too complicated to be parsed easily by \verb'scanf' so we
use the C standard library function \verb'strtok'. Each call of
\verb'strtok(p, s)' returns a pointer to the first token within \verb'p'
consisting of characters not in \verb's'; \verb'strtok' terminates the
token by overwriting the following character of the original string with a
null byte. On the first call, \verb'strtok''s first argument is the string
to scan; subsequent calls use \verb'NULL' to indicate that scanning should
resume where it left off (停止) in the previous call.  This is a poor
interface. Because \verb'strtok' stores a variable in a secret place
between calls, only one sequence of calls may be active at one time;
unrelated interleaved (交错的) calls will interfere with each other.

Our function \verb'unquote' removes the leading and trailing quotes that
appear in the sample input above. It does not handle nested quotes,
however, so although sufficient for a prototype, it's not general.
\begin{badcode}
    /* unquote: remove leading and trailing quote */
    char *unquote(char *p)
    {
        if (p[0] == '"') {
            if (p[strlen(p)-1] == '"')
                p[strlen(p)-1] = '\0';
            p++;
        }
        return p;
    }
\end{badcode}

A simple test program helps verify that \verb'csvgetline' works:
\begin{badcode}
    /* csvtest main: test csvgetline function */
    int main(void)
    {
        int i, nf;

        while ((nf = csvgetline(stdin)) != -1)
            for (i = 0; i < nf; i++)
                printf("field[%d] = '%s'\n", i, field[i]);;
            return 0;
    }
\end{badcode}
The \verb'printf' encloses the fields in matching single quotes, which
demarcate (区分) them and
help to reveal bugs that handle white space incorrectly.

We can now run this on the output produced by \verb'getquotes.tcl':
\begin{wellcode}
    % getquotes.tcl | csvtest
    ...
    field[0] = 'LU'
    field[1] = '86.375'
    field[2] = '11/5/1998'
    field[3] = '1:01PM'
    field[4] = '-0.125'
    field[5] = '86'
    field[6] = '86.375'
    field[7] = '85.0625'
    field[8] = '2888600'
    field[0] = 'T'
    field[1] = '61.0625'
    ...
\end{wellcode}
(We have edited out the HTTP header lines.)

Now we have a prototype that seems to work on data of the sort we showed
above.  But it might be prudent (谨慎的) to try it on something else as
well, especially if we plan to let others use it. We found another web site
that downloads stock quotes and obtained a file of similar information but
in a different form: carriage returns (\verb'\r') rather than newlines to
separate records, and no terminating carriage return at the end of the
file.  We've edited and formatted it to fit on the page:
\begin{wellcode}
    "Ticker","Price","Change","Open","Prev Close","Day High",
        "Day Low","52 Week High","52 Week Low","Dividend",
        "Yield","Volume","Average Volume","P/E"
    "LU",86.313,-0.188,86.000,86.500,86.438,85.063,108.50,
        36.18,0.16,0.1,2946700,9675000,N/A
    "T",61,125,0.938,60.375,60.188,61.125,60.000,68.50,
        46.50,1.32,2.1,3061000,4777000,17.0
    "MSFT",107.000,1.500,105.313,105.500,107.188,105.250,
        119.62,59.00,N/A,N/A,7977300,16965000,51.0
\end{wellcode}
With this input, our prototype failed miserably (非常不幸地).

We designed our prototype after examining one data source, and we tested it
originally only on data from that same source. Thus we shouldn't be
surprised when the first encounter with a different source reveals gross
(恶劣的) failings. Long input lines, many fields, and unexpected or missing
separators all cause trouble. This fragile prototype might serve for
personal use or to demonstrate the feasibility of an approach, but no more
than that. It's time to rethink the design before we try another
implementation.

We made a large number of decisions, both implicit and explicit, in the
prototype. Here are some of the choices that were made, not always in the
best way for a general-purpose library. Each raises an issue that needs
more careful attention.
\begin{itemize}
    \item The prototype doesn't handle long input lines or lots of fields.
        It can give wrong answers or crash because it doesn't even check
        for overflows, let alone return sensible values in case of errors.
    \item The input is assumed to consist of lines terminated by newlines.
    \item Fields are separated by commas and surrounding quotes are
        removed. There is no provision (准备) for embedded quotes or
        commas.  The input line is not preserved; it is overwritten by the
        process of creating fields.
    \item No data is saved from one input line to the next: if something is
        to be remembered, a copy must be made.
    \item Access to the fields is through a global variable, the
        \verb'field' array, which is shared by \verb'csvgetline' and
        functions that call it; there is no control over access to the
        \verb'field' contents or the pointers. There is also no attempt to
        prevent access beyond the last field.
    \item The global variables make the design unsuitable for a
        multi-threaded environment or even for two sequences of interleaved
        calls.
    \item The caller must open and close files explicitly;
        \verb'csvgetline' reads only from open files.
    \item Input and splitting are inextricably (无法避免地) linked: each
        call reads a line and splits it into fields, regardless of whether
        the application needs that service.
    \item The return value is the number of fields on the line; each line
        must be split to compute this value. There is also no way to
        distinguish errors from end of file.
    \item There is no way to change any of these properties without
        changing the code.
\end{itemize}

This long yet incomplete list illustrates some of the possible design
tradeoffs.  Each decision is woven through the code. That's fine for a
quick job, like parsing one fixed format from a known source. But what if
the format changes, or a comma appears within a quoted string, or the
server produces a long line or a lot of fields?

It may seem easy to cope (应付), since the "library" is small and only a
prototype anyway. Imagine, however, that after sitting on the shelf
(被搁置的) for a few months or years the code becomes part of a larger
program whose specification changes over time. How will \verb'csvgetline'
adapt? If that program is used by others, the quick choices made in the
original design may spell (招致) trouble that surfaces years later. This
scenario (情节) is representative of the history of many bad interfaces. It
is a sad fact that a lot of quick and dirty code ends up in widely-used
software, where it remains dirty and often not as quick as it should have
been anyway.
